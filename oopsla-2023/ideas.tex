%% -----------------------------------------------------------------------------

When a performance-debugging scenario arises, the key question is \emph{how
to modify the program} to improve performance.
Profiling tools provide data, but there are many ways to interpret this data.
The rational programmer method proceeds by enumerating possible
interpretations and testing each one independently.

To begin, the type-migration lattice suggests two general ways to modify a
code base:
add types to an untyped component, or toggle the types of a typed one
from deep to shallow or vice versa.
The next question is which component to modify.
Since profiling tools identify parts of the code base that contribute to performance
degradation, the logical choice is to rank them using a relevant, deterministic
order and modify the highest-priority one.

Stepping back, these two insights on modifications and ordering suggest
an experiment to determine which combinations of profiling tool,
ordering, and modification strategy help developers make progress with
performance debugging.
To determine the best combination(s), developers must work through a
large and diverse set of performance-debugging scenarios.
The result should identify successful and unsuccessful strategies for ranking
profiler output and modifying code.
Of course, it is unrealistic to ask human developers to follow faithfully different
strategies through thousands
of scenarios. An alternative experimental method is needed.

The rational programmer provides a framework for conducting such
large-scale systematic examinations.
It is inspired by the well-established idea of rationality in
economics~\cite{mill1874essays, henrich2001search}.  In more detail, a
rational agent is a mathematical model of an economic actor. Essentially, it
abstracts an actual economic actor to an entity that, in any given
transaction-scenario, acts strategically to maximize some kind of benefit.
These agents are (typically) bounded rather than perfectly rational
to reflect the limitations of human beings and of available information;
they aim to \emph{satisfice}~\cite{hs:satisfice} their goal since
they cannot make maximally optimal choices.
Analogously, a rational programmer is a model of a developer who aims to
resolve problems with bounded resources.
Specifically, it is an algorithm that implements a developer's bounded
strategy for {satisficing} a goal, and thereby enables a large-scale
experiment.
Developers can use the outcomes of an experiment to decide whether
``rational'' behavior seems to pay off.
In other words, a rational programmer evaluation yields insights into the
pragmatic value of work strategies.

So far, the rational programmer has been used to evaluate
strategies for debugging logical mistakes.\footnote{Prior work
distinguishes between \emph{strategies} for interpreting data and
\emph{modes} of the rational programmer, which combine a strategy and other
parameters into an algorithm. Our experiment has only one parameter, the
strategy, and therefore the distinction between strategy and mode is
unimportant here.}
This paper presents the first application to a performance problem.

%% -----------------------------------------------------------------------------
\paragraph{Experiment Sketch.}
In the context of profiler-guided
type migration, a rational programmer consists of two interacting pieces.  The
first is strategy-agnostic; it consumes a program, measures its running time,
and if the performance is tolerable, stops. Otherwise, the program is a
performance-debugging scenario and the second,
strategy-specific piece comes into play. This second piece profiles the given program---using
the boundary profiler or the statistical profiler---and analyzes the
profiling information. Based on this analysis, it modifies the program
as described above.
This modified version is handed back
to the first piece of the rational programmer.

There are many strategies that might prove useful.
A successful strategy will tend to eliminate performance overhead,
though perhaps after a few incremental steps.
An unsuccessful strategy will either degrade performance, or fail
to reach a fast configuration.
Testing several strategies sheds light on their relative usefulness.
If one strategy succeeds where another fails, it has higher relative value.
Of course, the experiment may also reveal shortcomings of the profiling approach
altogether---which would demand additional research from tool creators.

