%% -----------------------------------------------------------------------------

When a developer confronts a newly created performance-debugging scenario, one
possible response is to erase the types and to return to the {\em status quo
ante\/}.  Another one is to use a profiling tool to reduce the overhead. If the
second alternative is chosen, the question arises 
\begin{quote} \em

 how the developer should modify the program to obtain a version with tolerable
 performance.
 
\end{quote}   
The type-migration lattice suggests a straightforward answer.  The developer
should either (i) add types to an untyped component or (ii) experiment with
toggling existing types from deep to shallow protection and vice versa.  After
all, the fully typed program with deep run-time checks is known to be at least
as performant as its untyped counterpart, and therefore migration is guaranteed
to eliminate any performance bottlenecks sooner or later. 

Deciding which of the alternatives to pursue in a given situation poses the
follow-up question 
\begin{quote} \em

\item how a developer should interpret feedback from a profiling tool. 

\end{quote}   
Operationally, this question asks how the feedback is turned into a
modification at each step of the debugging process.

In contrast to the first question, the second one does not have a
straightforward answer.  At each step, the developer has to choose a
specific component to modify.  The sequence of choices should result in a
migratory path along which performance monotonically improves. Assuming
that the chosen profiling tool identifies components that contribute to
a performance bottleneck, it is rational for the programmer to rely on
its feedback to narrow down the set of candidate components.  Obtaining
profiler feedback is just the first step, however. Next the developer must
use the feedback information to rank the components and to identify the
highest-priority one for adding new types or toggling its existing types.
Abstracting from these answers, the programmer must choose a
\emph{profiling strategy}, that is a tool---statistical or
feature-specific---and an interpretation of its feedback that leads to a
program  modification. In this light, the answer to the question 
is a systematic examination of the
(relative) success of different profiling strategies in guiding a
developer who toggles or adds types to resolve a performance-debugging
scenario.

The rational programmer method provides a framework for conducting such a
systematic examination. The method is inspired by the well-established idea of
rationality in economics~\cite{mill1874essays, henrich2001search}.  In more
detail, a (bounded) rational agent is a mathematical model of an economic
actor. Essentially, it abstracts an actual economic actor to a formal entity that, in
any given transaction-scenario, acts strategically to maximize some kind of
benefit.  Of course, the strategy cannot be fully rational given the bounded
nature of human beings and the limits in available information. Hence,
investigations in modern times use rational agents that are bounded: they do not
make maximally optimal choices and settle
for \emph{satisficing}~\cite{hs:satisfice} their goal.

Analogously, a rational programmer is a model of a developer who aims to resolve
a scenario in a work-context, such as performance debugging, with bounded resources. In
specific terms, it is an algorithm that implements a developer's bounded
strategy for \emph{satisficing} its goal.  Hence, the successes and failures of
a rational programmer can inform human programmers about the success of
its strategy in the given work-context, i.e., the pragmatic value of the
strategy.




\bigskip

%% -----------------------------------------------------------------------------
\noindent\textbf{An Experiment Sketch.} In the context of profiler-guided
type migration, a rational programmer consists of two interacting pieces.
The first is strategy-agnostic; it consumes a performance-debugging
scenario, measures its running time, and if the performance is below an
acceptable level, asks for guidance from the second piece that is
strategy-specific. The strategy-specific piece uses either the feature or
the statistical profiler, obtains profiling feedback about the scenario,
and analyzes it. Then it makes a recommendation to the strategy-agnostic
piece about how to modify the scenario. The modification either resolves
the scenario's bottleneck, or produces a new, hopefully improved performance-wise,
scenario for the rational programmer to investigate further. 

In fact, as the discussion implies, there are numerous versions of  the
rational programmer that differ in their strategy-specific piece. Each
version, dubbed a \emph{mode}, can run on realistic performance-debugging
scenarios and apply its specific strategy to resolve their bottlenecks.
For any given scenario, after possibly multiple steps of modification, the
mode (and its strategy) either succeeds in eliminating the bottleneck, or
fails because the strategy makes the bottleneck worse or does not produce
any modification guidance. 

Data about successes and failures for different modes of the rational
programmer can shed light into the usefulness of one profiling strategy
compared to another.  Intuitively, the success rate of a mode of the
rational programmer (in)validates the hypothesis that its strategy has
pragmatic value. Similarly, the comparison of the  success rates of two
different modes on the same scenaria clarifies the relative pragmatic
value of their two strategies. Collecting the data necessary for drawing
reliable conclusions calls for an experiment that pits a spread
of strategies against each other on a large and diverse set of
performance-debugging scenaria.




