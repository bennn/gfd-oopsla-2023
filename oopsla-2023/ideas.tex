%% -----------------------------------------------------------------------------

When a developer confronts a newly created performance-debugging scenario, one
possible response is to erase the types and to return to the {\em status quo
ante\/}.  Another one is to use a profiling tool to reduce the overhead. If the
second alternative is chosen, the question arises 
\begin{quote} \em

 how the developer should modify the program to obtain tolerable performance.
 
\end{quote}   
The type-migration lattice suggests two answers.  The developer should either
add types to an untyped component or toggle the run-time checks for a typed one
from deep to shallow or vice versa.  After all, the fully typed program with
deep run-time checks is known to be at least as performant as its untyped
counterpart, and therefore migration is guaranteed to eliminate any performance
bottlenecks sooner or later.

Deciding which of the alternatives to pursue in a given situation poses the
follow-up question 
\begin{quote} \em

how a developer should interpret the feedback from a profiling tool to
choose a modification.

\end{quote}   
Operationally, this question asks how the feedback is turned into a
modification at each step of the performance-debugging process.

Unlike the first question, this second one does not have a simple answer.  At
each step, the developer has to choose a specific component to modify.  The
sequence of choices should result in a migratory path along which performance
improves. Assuming that the chosen profiling tool identifies components that
contribute to the performance degradation, it is rational for the programmer to
rely on this feedback to rank the components and to identify the
highest-priority one for adding types or for toggling its run-time checking.

Stepping back, these two insights suggest an experiment to determine which
profiling tool combined with which strategy helps developers make progress with
performance debugging. For this experiment, the developer must choose a
profiling tool---statistical or feature-specific---and a modification strategy.
To determine the best combination(s) means to have developers work through a
large number of performance-debugging scenarios. The result should help guide a
developer with toggling or adding types to resolve performance-related
problems during type migration.

Since it is obviously impossible to ask a developer to work through thousands of
performance-debugging scenarios, an alternative experimental method is needed.
The rational-programmer method provides a framework for conducting such
large-scale systematic examinations.

The method is inspired by the well-established idea of rationality in
economics~\cite{mill1874essays, henrich2001search}.  In more detail, a (bounded)
rational agent is a mathematical model of an economic actor. Essentially, it
abstracts an actual economic actor to an entity that, in any given
transaction-scenario, acts strategically to maximize some kind of benefit.  Of
course, the strategy cannot be fully rational given the bounded nature of human
beings and the limits in available information. Hence, investigations in modern
times use rational agents that are bounded; they do not make maximally optimal
choices and settle for \emph{satisficing}~\cite{hs:satisfice} their goal.

Analogously, a rational programmer is a model of a developer who aims to resolve
a problem in a work context---a scenario---with bounded
resources. Specifically, it is an algorithm that implements a developer's
bounded strategy for \emph{satisficing} its goal. Using such algorithms, it is
possible to inspect the effectiveness of an approach on a large quantity of
scenarios. In turn, comparing the results of such runs for different rational
programmers can inform human programmers about the success chances of a strategy
in some specific work context. In short, a rational-programmer evaluation yields
insights into the pragmatic value of work strategies. So far, the rational
programmer has been used to evaluate strategies for debugging logical mistakes;
this paper presents the first application to a performance problem. 

\medskip

%% -----------------------------------------------------------------------------
\paragraph{A sketch of a rational-programmer experiment for performance
debugging migratory typing overhead problems.}  In the context of profiler-guided
type migration, a rational programmer consists of two interacting pieces.  The
first is strategy-agnostic; it consumes a program, measures its running time,
and if the performance is tolerable, stops. Otherwise it faces a
performance-debugging scenario and asks for guidance from the second,
strategy-specific piece. This second piece profiles the given program---using
the feature-specific profiler or the statistical profiler---and analyzes the
profiling information. Based on this analysis, it modifies the program by adding
types to some untyped component to eliminate a boundary or toggling the run-time
enforcement of types for a typed component. This modified version is handed back
to the first piece of the rational programmer that repeats the process if the
program is more performant after the modification.  Otherwise the rational
programmer stops.

In fact, as the discussion implies, there are numerous versions of the rational
programmer that differ in their strategy-specific piece. Each such version,
dubbed a \emph{mode}, can be applied to realistic performance-debugging
scenarios and use its specific strategy to resolve the bottleneck.  For any
given scenario, the mode either succeeds in eliminating the bottleneck or fails
because the strategy cannot propose a program modification or because the chosen
modification worsens the performance of the program. 

Data about successes and failures for different modes can shed light on the
usefulness of one profiling strategy compared to another. Intuitively, the
success rate of a mode (in)validates the hypothesis that its strategy has
pragmatic value. Similarly, the comparison of the success rates of two different
modes on the same scenario clarifies the relative pragmatic value of the
two. Collecting the data necessary for drawing reliable conclusions calls for an
experiment that pits a spread of strategies against each other on a large and
diverse set of performance-debugging scenarios. Of course, the experiment may
also reveal shortcomings of the profiling approach altogether---which then
demands additional research from tool creators. 
