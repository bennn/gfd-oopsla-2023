%% -----------------------------------------------------------------------------
Every program \program{} is a collection of interacting components $\component$.
Some components have deep types, some shallow ones, and some
are untyped. Independently of their types, a component $\component{}_1$, may
import another component $\component{}_2$, which establishes a \emph{boundary}
between them, across which they exchange values at run time. Depending on the
kind of types at the two sides of the boundary, a value exchange can trigger
run-time checks, which may degrade the performance.

A profiling strategy should thus aim to eliminate the most costly
boundaries in a program. In formal terms, a profiling strategy is a
function that consumes a program \program{} and, after determining its
profile, returns a set of pairs $(\component{}, \type{})$. Here \type{} is
either \deep{} or \shallow{}. Each such pair prescribes a modification of
\program{}. For instance, if a strategy returns the singleton set with the
pair $(c, \deep)$, then the strategy points to a new version of \program{}
where component $c$ obtains deep types (if necessary) ; if $c$ is typed,
the strategy just requests toggling from shallow to deep.  If a strategy's
result is the empty set, it cannot figure out how to proceed.

%% -----------------------------------------------------------------------------

\def\with{with {\em total\/} in place of {\em self\/}}

\begin{figure}[htb]

  \newcommand{\desc}[1]{\parbox[t]{24em}{#1\\[-2mm]}}
  
  \def\desca{\desc{Uses the boundary profiler to identify the most
    expensive boundary in the given program. Then it directs the mode to equip
    both sides of the target boundary with deep types.}}

  \def\descb{\desc{Like \featopt{} but with shallow types for both
    sides of the target boundary.}}

  \def\descc{\desc{Uses the statistical profiler to identify the component
    $\component{}_1$ with the highest self time in the given program that has a
    boundary with at least one component $\component{}_2$ that has stricter
    types than $\component{}_1$.  Then it picks the component $\component{}_2$
    that has the highest self time, and directs the mode to equip both
    $\component{}_1$ and $\component{}_2$ with deep types.}}

  \def\descd{\desc{Like \statselfopt{} \with}}

  \def\desce{\desc{Like \statselfopt{}, with shallow types for $\component{}_1$, $\component{}_2$}}

  \def\descf{\desc{Like \statselfcon{} with {\em total\/} in place of {\em self\/}}}

 \begin{tabular}{l l l}
    {\bf Profiler} & {\bf Strategy} & {\bf Description} \\ \hline
    %%%%%%%%%%%%%%%%%%%%%%%%%%%%%%%%%%%%%%%%%%%%%%%%%%%%%%%%%%%%%%%%
    \multirow[b]{2}[+5]{*}{{\em boundary\/}} & 
        \optkw{}          &   \desca          \\ \relax
     &	\conkw{}          &   \descb          \\ \hline

    \multirow[b]{2}[+7]{*}{{\em statistically\/} ({\em self\/})} &
	\optkw{}      &   \descc          \\ \relax
     &	\conkw{}      &   \desce          \\ \hline

    \multirow{2}{*}{{\em statistically\/} ({\em total\/})} & 
	\conkw{}     &   \descf           \\ \relax
     &	\optkw{}     &   \descd           % \\ \relax
 \end{tabular}

 
  \caption{The Basic Strategies of the Experiment and their Descriptions}
  \label{f:bstrategies}
\end{figure}
%% -----------------------------------------------------------------------------


\paragraph{Basic strategies.}  Figure~\ref{f:bstrategies} describes six basic
 strategies that rational programmers may use. 

At the first level, the basic strategies differ in whether they employ the
 \featkw{} or the \statkw{} profiler to profile the given program.  Those that
 use the first kind of profiler can directly identify the most costly boundary
 in \program{} and suggest a modification of either of the two components.
 Those that use the second cannot directly identify a boundary.  Instead they
 focus on the most costly component $\component{}_1$ in terms of either
 \selfkw{} or \totalkw{} time.  From there, they pick one of the boundaries
 between $\component{}_1$ and any of the components that it depends on or that
 depend on it. The goal is to find a boundary between $\component{}_1$ and a
 component $\component{}_2$ where $\component{}_2$ has \emph{stricter} types
 than $\component{}_1$, because in this case, the interactions across the
 boundary are likely to generate costly run-time type checks. Here, deep is
 stricter than shallow, and shallow is stricter than untyped. If the strategy
 cannot identify such a boundary, it moves on to the next costly component
 (again in terms of either \selfkw{} or \totalkw{} time). Once it eventually
 identifies a target boundary, the strategy decides how to migrate the
 two components.

At the second level, basic strategies differ in how they migrate the two sides
 of their target boundary. Strategies that are \optkw{} turn the types at either
 side of the boundary to deep. After all, when two components with deep types
 interact, no checks take place and, better still, deep enforcement may even
 enable type-driven compiler optimizations for the two components.  Hence, such
 a migration may eliminate the cost of the boundary entirely.  Experience shows,
 however, that switching to deep type enforcement for a component may cause a
 ripple effect. The new deeply enforced types may increase the cost of other
 boundaries between (one of) the two components and a third one. By contrast,
 \conkw{} strategies choose shallow enforcement of types for both sides of the
 target boundary. The rationale behind this choice is that, if both sides of a
 boundary have shallow types, the interactions across the boundary cost less
 than if only one is deep and, at the same time, unlike with \optkw{}
 strategies, there is no risk of a ripple effect.

%% -----------------------------------------------------------------------------
\begin{figure}[b]
 
  \newcommand{\desc}[1]{\parbox[t]{19.5em}{\raggedright #1\\[-2mm]}}
 
  \def\desca{\desc{Splits the boundaries in the given program to those between
    typed components and the rest.  Delegates to \featopt{} to produce a
    modification for the given program, but rank boundaries in the first group
    higher than those in the second group while determining the most expensive
    boundary.}}

  \def\descb{\desc{Like \featcostopt{} but it delegates to \featcon{}.}}

  \def\descc{\desc{Separates the typed components that have boundaries with
    other typed components from the rest of the components in the given
    program. Delegates to \statselfopt{} to produce a modification for the given
    program, but rank boundaries between components in the first group higher
    than the rest while determining the most expensive boundary.}}

   \def\descd{\desc{Like \statselfcostopt{} but it delegates to \stattotalopt{}.}}

   \def\desce{\desc{Like \statselfcostopt{} but it delegates to \statselfcon{}.}}

   \def\descf{\desc{Like \statselfcostopt{} but it delegates to \stattotalcon{}.}}

   \def\descg{\desc{If the number of typed components in the given program is
     above a threshold $N$, it delegates to \featopt{}. Otherwise, it delegates
     to \featcon{}.}}

   \def\desch{\desc{Like \featconf{}; it delegates to \statselfopt{}
     or \statselfcon{} instead.}}

  \def\descj{\desc{Like \featconf{}; it delegates to \stattotalopt{}
     or \stattotalcon{} instead.}}

 \begin{tabular}{l l l}
    {\bf Profiler} & {\bf Response} & {\bf Description}  \\ \hline
    %%%%%%%%%%%%%%%%%%%%%%%%%%%%%%%%%%%%%%%%%%%%%%%%%%%%%%%%%%%%%%%%
    \multirow[b]{2}[+19]{*}{{\em boundary\/}} & 
    \costoptkw{}      &   \desca           \\ \relax
    & \costconkw{}      &   \descb           \\ \relax
    & \confkw{}         &   \descg           \\ \hline

    \multirow[b]{2}[+17]{*}{{\em statistical\/} ({\em self\/})} &
    \costoptkw{}   &   \descc           \\ \relax
    & \costconkw{}   &   \desce           \\ \relax
    & \confkw{}      &   \desch           \\ \hline

    \multirow{2}[+17]{*}{{\em statistical\/} ({\em total\/})} & 
    \costoptkw{} &   \descd           \\ \relax
    & \costconkw{} &   \descf           \\ \relax
    & \confkw{}    &   \descj           % \\ \relax

 \end{tabular}

 
  \caption{The Composite Strategies of the Experiment and their Descriptions}
  \Description{The Composite Strategies of the Experiment and their Descriptions}
  \label{f:cstrategies}
\end{figure}
%% -----------------------------------------------------------------------------


\paragraph{Composite strategies.} While the basic strategies ignore the cost of
 adding types to an untyped components, developers do not. Adding types to an
 entire module in Typed Racket may require a significant effort. Hence, the
 experiment includes composite strategies that account for this cost.

Figure~\ref{f:cstrategies} lists these composite strategies. In addition to the
 performance cost reported in the profile, the \costkw{} strategies rank the
 cost of boundaries in terms of the labor needed to equip the two components
 with types.  They give priority to those boundaries that involve components
 that are already typed.  For those, migration just means toggling their type
 enforcement regime, which is essentially no labor.  As pointed out, ripple
 effects also factor into the decisions that strategies have to make. The
 \confkw{} strategies decide whether to be \optkw{} or \conkw{} based on how
 many components are already typed.  When most components are untyped, the risk
 of a ripple effect outweighs the benefits of an \optkw{} strategy. Hence
 strategies favor a \conkw{} approach for sparsely typed programs and an
 \optkw{} strategy for densely typed ones.

\paragraph{Baseline Strategies} An experiment must include baselines, i.e., the
 building block for a null hypothesis. In terms of strategies, the experiment
 needs a \agnostickw{} strategy.  If this \agnostickw{} strategy is less
 successful in eliminating performance problems than the profiling ones, then
 the feedback from the profiler plays a meaningful role; otherwise, the
 experiment is meaningless.

The results presented in the next section include two 
 \agnostickw{} strategies. The first one, \randkw{}, aims to invalidate the null
 hypothesis with random choices. Specifically, it picks a random boundary with
 types of different strictness and suggests the modification of the two sides of the
 target boundary in either an \optkw{} or a \conkw{} manner.  The second
 \agnostickw{} strategy, \togglekw{}, is due to~\citet{g-deep-shallow}. It
 serves as a point of comparison with Greenman's results, which do not rely on
 profiling information. If the given program uses a mixture of shallow and deep
 enforcement, the strategy recommends that all
 typed components toggle to deep enforcement. If all typed components already use deep, it
 recommends toggling the enforcement to shallow.
