%% -----------------------------------------------------------------------------
Every program \program{} is a collection of interacting components $\component$.
Some components have deep types, some have shallow ones, and some
are untyped. Independently of their types, a component $\component{}_1$, may
import another component $\component{}_2$, which establishes a \emph{boundary}
between them, across which they exchange values at run time. Depending on the
kind of types at the two sides of the boundary, a value exchange can trigger
run-time checks, which may degrade performance.

A profiling strategy should thus aim to eliminate the most costly
checks in a program. In formal terms, a profiling strategy is a
function that consumes a program \program{} and, after determining its
profile, returns a set of pairs $(\component{}, \type{})$. Here \type{} is
either \deep{} or \shallow{}. Each such pair prescribes a modification of
\program{}. For instance, if a strategy returns the singleton set with the
pair $(c, \deep)$, then the strategy points to a new version of \program{}
where component $c$ obtains deep types (if necessary) ; if $c$ is typed,
the strategy just requests toggling from shallow to deep.  If a strategy's
result is the empty set, it cannot figure out how to proceed.

\paragraph{Basic strategies.}  \Cref{f:bstrategies} describes six basic
strategies that rational programmers may use.
The strategies differ along two levels: how to use profiler data to identify
a set of checks and how to modify the program toward lower costs.

At the first level, the basic strategies choose a profiler and (when
necessary) an ordering for its output.  The profiler is either \featkw{}
or \statkw{}~(\cref{sec:seascape}).  With the \boundary{} profiler, the
output is a list of boundaries ordered by cost, so there is no need for
the rational programmer to choose an ordering.  With the \statistical{}
profiler, the output is a list of applications each with two types of costs:
the \totalkw{} time spent during the call including its
dependencies, and the \selfkw{} time spent in the call not including dependencies.
Because both costs are potentially useful, the rational programmers choose
between them.  Having ordered the applications, these rational programmers
must then identify a boundary.  They start with the top-ranked application 
and seek a boundary between the enclosing component and a component with
\emph{stricter} types because the types at those boundaries incur run-time checks.
Here, deep is stricter than shallow and shallow is stricter than untyped.
If the strategy cannot identify such a boundary, it moves to the
next-ranked application (again in terms of either \selfkw{} or \totalkw{}
time).  If there are no applications remaining, the strategy fails.

%% -----------------------------------------------------------------------------

\def\with{with {\em total\/} in place of {\em self\/}}

\begin{figure}[htb]

  \newcommand{\desc}[1]{\parbox[t]{24em}{#1\\[-2mm]}}
  
  \def\desca{\desc{Uses the boundary profiler to identify the most
    expensive boundary in the given program. Then it directs the mode to equip
    both sides of the target boundary with deep types.}}

  \def\descb{\desc{Like \featopt{} but with shallow types for both
    sides of the target boundary.}}

  \def\descc{\desc{Uses the statistical profiler to identify the component
    $\component{}_1$ with the highest self time in the given program that has a
    boundary with at least one component $\component{}_2$ that has stricter
    types than $\component{}_1$.  Then it picks the component $\component{}_2$
    that has the highest self time, and directs the mode to equip both
    $\component{}_1$ and $\component{}_2$ with deep types.}}

  \def\descd{\desc{Like \statselfopt{} \with}}

  \def\desce{\desc{Like \statselfopt{}, with shallow types for $\component{}_1$, $\component{}_2$}}

  \def\descf{\desc{Like \statselfcon{} with {\em total\/} in place of {\em self\/}}}

 \begin{tabular}{l l l}
    {\bf Profiler} & {\bf Strategy} & {\bf Description} \\ \hline
    %%%%%%%%%%%%%%%%%%%%%%%%%%%%%%%%%%%%%%%%%%%%%%%%%%%%%%%%%%%%%%%%
    \multirow[b]{2}[+5]{*}{{\em boundary\/}} & 
        \optkw{}          &   \desca          \\ \relax
     &	\conkw{}          &   \descb          \\ \hline

    \multirow[b]{2}[+7]{*}{{\em statistically\/} ({\em self\/})} &
	\optkw{}      &   \descc          \\ \relax
     &	\conkw{}      &   \desce          \\ \hline

    \multirow{2}{*}{{\em statistically\/} ({\em total\/})} & 
	\conkw{}     &   \descf           \\ \relax
     &	\optkw{}     &   \descd           % \\ \relax
 \end{tabular}

 
  \caption{The Basic Strategies of the Experiment and their Descriptions}
  \label{f:bstrategies}
\end{figure}
%% -----------------------------------------------------------------------------


At the second level, basic strategies differ in how they migrate the two sides
 of their target boundary. Strategies that are \optkw{} turn the types at either
 side of the boundary to deep.
 This action eliminates the cost of the boundary and enables type-driven
 optimizations in both components.
 But, it may also create boundaries to other components in a kind of ripple
 effect with potentially disastrous costs.
 By contrast, \conkw{} strategies choose shallow types for both
 sides of the target boundary.
 The rationale behind this choice is that, if both sides of a
 boundary have shallow types, the interactions across the boundary cost less
 than if only one is deep and, at the same time, unlike with \optkw{}
 strategies, there is no risk of a ripple effect.

%% -----------------------------------------------------------------------------
\begin{figure}[b]
 
  \newcommand{\desc}[1]{\parbox[t]{19.5em}{\raggedright #1\\[-2mm]}}
 
  \def\desca{\desc{Splits the boundaries in the given program to those between
    typed components and the rest.  Delegates to \featopt{} to produce a
    modification for the given program, but rank boundaries in the first group
    higher than those in the second group while determining the most expensive
    boundary.}}

  \def\descb{\desc{Like \featcostopt{} but it delegates to \featcon{}.}}

  \def\descc{\desc{Separates the typed components that have boundaries with
    other typed components from the rest of the components in the given
    program. Delegates to \statselfopt{} to produce a modification for the given
    program, but rank boundaries between components in the first group higher
    than the rest while determining the most expensive boundary.}}

   \def\descd{\desc{Like \statselfcostopt{} but it delegates to \stattotalopt{}.}}

   \def\desce{\desc{Like \statselfcostopt{} but it delegates to \statselfcon{}.}}

   \def\descf{\desc{Like \statselfcostopt{} but it delegates to \stattotalcon{}.}}

   \def\descg{\desc{If the number of typed components in the given program is
     above a threshold $N$, it delegates to \featopt{}. Otherwise, it delegates
     to \featcon{}.}}

   \def\desch{\desc{Like \featconf{}; it delegates to \statselfopt{}
     or \statselfcon{} instead.}}

  \def\descj{\desc{Like \featconf{}; it delegates to \stattotalopt{}
     or \stattotalcon{} instead.}}

 \begin{tabular}{l l l}
    {\bf Profiler} & {\bf Response} & {\bf Description}  \\ \hline
    %%%%%%%%%%%%%%%%%%%%%%%%%%%%%%%%%%%%%%%%%%%%%%%%%%%%%%%%%%%%%%%%
    \multirow[b]{2}[+19]{*}{{\em boundary\/}} & 
    \costoptkw{}      &   \desca           \\ \relax
    & \costconkw{}      &   \descb           \\ \relax
    & \confkw{}         &   \descg           \\ \hline

    \multirow[b]{2}[+17]{*}{{\em statistical\/} ({\em self\/})} &
    \costoptkw{}   &   \descc           \\ \relax
    & \costconkw{}   &   \desce           \\ \relax
    & \confkw{}      &   \desch           \\ \hline

    \multirow{2}[+17]{*}{{\em statistical\/} ({\em total\/})} & 
    \costoptkw{} &   \descd           \\ \relax
    & \costconkw{} &   \descf           \\ \relax
    & \confkw{}    &   \descj           % \\ \relax

 \end{tabular}

 
  \caption{The Composite Strategies of the Experiment and their Descriptions}
  \Description{The Composite Strategies of the Experiment and their Descriptions}
  \label{f:cstrategies}
\end{figure}
%% -----------------------------------------------------------------------------


\paragraph{Composite strategies.} While the basic strategies ignore the cost of
 writing type annotations for an untyped component, developers do not.
 Adding types to an entire module in Typed Racket may require a significant
 effort.
 Similarly, the likelihood of ripple-effect costs depends on the number
 of typed components in the program.
 With few types, the cost of introducing one component with deep types may well be high;
 with many types, the chance of a ripple effect is probably low.
 Hence, the experiment includes composite strategies that take into account the
 types currently in the codebase before choosing how to respond to profile data.

\Cref{f:cstrategies} lists these composite strategies.
 The \costkw{} strategies rank the cost of boundaries in terms of the labor needed to equip the two components
 with types in addition to the costs reported by the profiler.
 They give priority to those boundaries that involve components
 that are already typed.  For those, migration just means toggling their type
 enforcement regime, which is essentially no labor.
 The \confkw{} strategies use a heuristic to avoid ripple effects.
 Instead of committing to a type-enforcement regime up front
 (optimistically or conservatively), they choose shallow when most components
 are untyped and deep when most are typed.

\paragraph{Baseline Strategies} An experiment must include a baseline, i.e., the
 building block for a null hypothesis.
 Since profilers are the focus of this experiment, baselines must be
 \agnostickw{}.
 If strategies that ignore profiler data do worse than the basic and composite
 strategies, then feedback from the profiler evidently plays a meaningful role.
 Otherwise, comparisons among profiler-aware strategies are meaningless.

The results presented in the next section include two 
 \agnostickw{} strategies. The first one, \randkw{}, aims to invalidate the null
 hypothesis with random choices. Specifically, it picks a random boundary with
 types of different strictness and flips a coin to choose either an
 \optkw{} or a \conkw{} modification to both sides.
 The second \agnostickw{} strategy, \togglekw{}, is due to~\citet{g-deep-shallow}
 and serves as a point of comparison to that prior work.
 It modifies all typed components to use the same checks, deep or shallow,
 depending on which regime gives the best performance.
 It never adds types to an untyped component, which means this strategy has
 only one chance to improve performance.

