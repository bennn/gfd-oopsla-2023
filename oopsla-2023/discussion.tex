	
%% MF: no clue why this is here. Some of these are about plots, which
%% mostly belong into sec. 5. Some are about scatter plots, which we
%% abandoned. 

% TODO for scatterplot: x-axis = 1-20, but actual plot goes 1-30 and uses the
% last 10 to evenly spread the rest.

% 1. boundary vs profile total vs profile self, on the whole. scatterplot too,
%    boundary usually improves, profile frequently degrades

% 2. strategies: opt vs con vs .... who wins?

% 3. random is very successful b/c no dead ends.
%    why do the profilers have dead ends?

% 4. (maybe) implictions of 0-dip, 1-dip, 3-dip, N-dip,
%    rare that N improves over 3

% 5. non-overlapping success / failure.
%    if you give us a configuration, can we suggest a best profiler / strategy?
%    all-shallow probably fails with boundary, etc

% 6. head to head plot:
%    - opts equiv
%    - con never beats opt
%    - statistical rarely beats fsm, but sometimes!

% 7. why does random succeed N-loose when profiles fail? profiles hit dead ends


%% -----------------------------------------------------------------------------

% How successful is a strategy $X$ with the elimination of performance bottlenecks?

$Q_X$

- level 1: any optimistic or cost-aware strategy is more likely to
  succeed than a conservative one; indeed, random and toggling are
  more likely to succeed than conservative one

- level 2: when feedback is successful, the feature-specific profiler
  is far superior to all others. That's true in general as well as 

- level 1: interpreting the feedback from the feature-specific
  profiler is more likely to succeed than 

%% -----------------------------------------------------------------------------

% Is strategy $X$ more successful than strategy $Y$ in this context?

$Q_{X/Y}$

%% -----------------------------------------------------------------------------
\subsection{Threats to Validity}


``black holes''
