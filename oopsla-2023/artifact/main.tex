\documentclass{article}

\usepackage{colortbl}
\usepackage{graphicx}
\usepackage{xcolor}

\newcommand{\bmname}[1]{\textsf{#1}}
\newcommand{\gcell}[1]{\cellcolor{green!20}#1}
\newcommand{\wcell}[1]{\cellcolor{black!05}#1}
\newcommand{\ycell}[1]{\cellcolor{yellow!18}#1}
\newcommand{\ocell}[1]{\cellcolor{orange!29}#1}
\newcommand{\rcell}[1]{\cellcolor{red!30}#1}
\newcommand{\tcell}[1]{\cellcolor{black!10}#1}

\begin{document}

\begin{table}[ht]\centering
  \caption{How many configurations have any overhead to begin with?}
  \label{t:baseline-trouble}
  \begin{tabular}{lrr}
    Benchmark & $3^N$ & \% Slow \\
\bmname{morsecode} & 81 & 82.72\% \\
\bmname{forth} & 81 & 93.83\% \\
\bmname{fsm} & 81 & 76.54\% \\
\bmname{fsmoo} & 81 & 83.95\% \\
\bmname{mbta} & 81 & 88.89\% \\
\bmname{zombie} & 81 & 91.36\% \\
\bmname{dungeon} & 243 & 99.59\% \\
\bmname{jpeg} & 243 & 94.65\% \\
\bmname{lnm} & 729 & 40.47\% \\
\bmname{suffixtree} & 729 & 98.49\% \\
\bmname{kcfa} & 2187 & 92.87\% \\
\bmname{snake} & 6561 & 99.97\% \\
\bmname{take5} & 6561 & 99.95\% \\
\bmname{acquire} & 19683 & 99.23\% \\
\bmname{tetris} & 19683 & 95.47\% \\
\bmname{synth} & 59049 & 99.99\% \\

  \end{tabular}
\end{table}

\begin{figure}[ht]
  \includegraphics[width=\columnwidth]{tex/strategy-overall-feasible.pdf}
  \caption{How many scenarios does each strategy succeed in, for six notions of success.}
  \label{f:strategy-overall}
\end{figure}

\begin{table}[ht]\centering
  \caption{How many scenarios can possibly reach the goal without removing types?}
  \label{t:blackhole}
  \begin{tabular}{lrr}
    Benchmark & \# Scenario & \% Hopeful \\
\bmname{morsecode} & 67 & 100\% \\
\bmname{forth} & 76 & 36.84\% \\
\bmname{fsm} & 62 & 100\% \\
\bmname{fsmoo} & 68 & 100\% \\
\rcell{\bmname{mbta}} & \rcell{72} & \rcell{0\%} \\
\bmname{zombie} & 74 & 35.14\% \\
\rcell{\bmname{dungeon}} & \rcell{242} & \rcell{0\%} \\
\bmname{jpeg} & 230 & 100\% \\
\bmname{lnm} & 295 & 100\% \\
\bmname{suffixtree} & 718 & 100\% \\
\bmname{kcfa} & 2031 & 100\% \\
\bmname{snake} & 6559 & 100\% \\
\rcell{\bmname{take5}} & \rcell{6558} & \rcell{0\%} \\
\bmname{acquire} & 19532 & 5.45\% \\
\bmname{tetris} & 18791 & 100\% \\
\bmname{synth} & 59046 & 100\% \\

  \end{tabular}
\end{table}

\begin{figure}[ht]
  \includegraphics[width=\columnwidth]{tex/strategy-overall-hopeful.pdf}
  \caption{How many of the hopeful scenarios does each strategy succeed in, for six notions of success.}
  \label{f:strategy-hope}
\end{figure}

\begin{figure}[ht]
  \includegraphics[width=0.9\columnwidth]{tex/head-to-head.pdf}
  \caption{Boundary optimistic vs. the rest, strict success: losses (red bars) and wins (green bars) on all scenarios.}
  \label{f:head-to-head}
\end{figure}

\begin{figure}[ht] \footnotesize
  \newcommand{\kkrow}[1]{\bmname{#1} \\ \includegraphics[width=0.39\columnwidth]{tex/sky/#1-feasible.pdf}}
  \begin{tabular}[t]{ll}
    \begin{tabular}[t]{l}
      \kkrow{morsecode} \\
      \kkrow{forth} \\
      \kkrow{fsm} \\
      \kkrow{fsmoo} \\
      \kkrow{mbta} \\
      \kkrow{zombie} \\
      \kkrow{dungeon} \\
      \kkrow{jpeg} \\
    \end{tabular}
    &
    \begin{tabular}[t]{l}
      \kkrow{lnm} \\
      \kkrow{suffixtree} \\
      \kkrow{kcfa} \\
      \kkrow{snake} \\
      \kkrow{take5} \\
      \kkrow{acquire} \\
      \kkrow{tetris} \\
      \kkrow{synth}
    \end{tabular}
  \end{tabular}

  \caption{How scenarios in each benchmark does each strategy succeed in?}
  \label{f:skyline:bm}
\end{figure}

\begin{figure}[t] \footnotesize
  \newcommand{\hhrow}[1]{\bmname{#1} \\ \includegraphics[width=0.39\columnwidth]{tex/h2h/#1.pdf}}
  \begin{tabular}[t]{ll}
    \begin{tabular}[t]{l}
      \hhrow{morsecode} \\
      \hhrow{forth} \\
      \hhrow{fsm} \\
      \hhrow{fsmoo} \\
      \hhrow{mbta} \\
      \hhrow{zombie} \\
      \hhrow{dungeon} \\
      \hhrow{jpeg} \\
    \end{tabular}
    &
    \begin{tabular}[t]{l}
      \hhrow{lnm} \\
      \hhrow{suffixtree} \\
      \hhrow{kcfa} \\
      \hhrow{snake} \\
      \hhrow{take5} \\
      \hhrow{acquire} \\
      \hhrow{tetris} \\
      \hhrow{synth}
    \end{tabular}
  \end{tabular}

  \caption{Optimistic vs. the rest, comparing strict successes in each benchmark.}
  \label{f:h2h:bm}
\end{figure}
\end{document}
