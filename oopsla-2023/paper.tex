\documentclass[acmsmall,review,anonymous]{acmart}

%\settopmatter{printfolios=true,printccs=false,printacmref=false}


\overfullrule=3mm
\citestyle{acmauthoryear}
%\setcitestyle{round}

\usepackage{multirow}

\usepackage{alltt}
% \usepackage{amssymb}
\usepackage{calc}
\usepackage{cleveref}
\usepackage{colortbl}
\usepackage{listings}
\usepackage{mathpartir}
\usepackage{graphicx}
\usepackage{pifont}
\usepackage{subcaption}
\usepackage{tikz}
\usepackage{wrapfig}
\usepackage{xcolor}
\usetikzlibrary{arrows.meta}
\usetikzlibrary{positioning}
\usetikzlibrary{shapes.geometric}

\usepackage{moreverb}

%%%%%%%%%%%%%%%%%%%%%%%%%%%%%%%%%%%%%%%%%%%%%%%%%%%%%%%%%%%%%%%%%%%%%%%%%%%%%%%%%%%%%%%%%%%%
%%%%%%%%%%%%%%%%%%%%%%%%%%%%%%%%%%%%%%%%%%%%%%%%%%%%%%%%%%%%%%%%%%%%%%%%%%%%%%%%%%%%%%%%%%%%
%%%%%%%%%%%%%%%%%%%%%%%%%%%%%%           PREAMBLE       %%%%%%%%%%%%%%%%%%%%%%%%%%%%%%%%%%%% 
%%%%%%%%%%%%%%%%%%%%%%%%%%%%%%%%%%%%%%%%%%%%%%%%%%%%%%%%%%%%%%%%%%%%%%%%%%%%%%%%%%%%%%%%%%%%
%%%%%%%%%%%%%%%%%%%%%%%%%%%%%%%%%%%%%%%%%%%%%%%%%%%%%%%%%%%%%%%%%%%%%%%%%%%%%%%%%%%%%%%%%%%%

\usepackage{tcolorbox}

%%%%%%%%%%%%%%%%%%%%%%%%%%%%%%%%%%%%%%%%%%%%%%%%%%%%%%%%%%%%%%%%%%%%%%%%%%%%%%%%%%%%%%%%%%%%
%%%%%%%%%%%%%%%%%%%%%%%%%%%%%%%%%%%%%%%%%%%%%%%%%%%%%%%%%%%%%%%%%%%%%%%%%%%%%%%%%%%%%%%%%%%%
%%%%%%%%%%%%%%%%%%%%%%%%%%%%%%%%%%%%%%%%%%%%%%%%%%%%%%%%%%%%%%%%%%%%%%%%%%%%%%%%%%%%%%%%%%%%

\def\set#1{\ensuremath{\bar{#1}}}

\def\size#1{\ensuremath{\mid #1 \mid}}

\def\setsize#1{\left|#1\right|}



%% %%%%%%%%%%%%%%%%%%%%%%%%%%%%%%%%%%%%%%%%%%%%%%%%%%%%%%%%%%%%%%%%%%%%%%%%%%%%%
%% for leaving margin notes in the paper write
%% \yourinitials{...} 

\def\notes#1{\expandafter\def\csname#1\endcsname##1{\marginpar{\textcolor{red}{\raggedright\tiny $\bullet$ #1 says: ##1}}}}
\notes{mf}
\notes{cd}
\notes{bg}


%% %%%%%%%%%%%%%%%%%%%%%%%%%%%%%%%%%%%%%%%%%%%%%%%%%%%%%%%%%%%%%%%%%%%%%%%%%%%%%

\def\set#1{\ensuremath{\bar{#1}}}

\def\size#1{\ensuremath{\mid #1 \mid}}

\def\setsize#1{\left|#1\right|}


%% strategies

\newcommand{\sfont}[1]{\textit{#1}}

\newcommand{\featkw}{\sfont{feature\textsf{-}specific}}

\newcommand{\statkw}{\sfont{statistical}}

\newcommand{\totalkw}{\sfont{total}}

\newcommand{\selfkw}{\sfont{self}}

\newcommand{\stat}[1]{\sfont{\statkw(#1)}}

\newcommand{\statself}{\sfont{\stat{\selfkw}}}
\newcommand{\stattotal}{\sfont{\stat{\totalkw}}}

\newcommand{\optkw}{\sfont{optimistic}}

\newcommand{\conkw}{\sfont{conservative}}

\newcommand{\costkw}{\sfont{cost\textsf{-}aware}}

\newcommand{\confkw}{\sfont{configuration\textsf{-}aware}}

\newcommand{\randkw}{\sfont{null}}

\newcommand{\togglekw}{\sfont{either-or}}

\newcommand{\agnostickw}{\sfont{profiler\textsf{-}agnostic}}

\newcommand{\strategy}[2]{#1~#2}

\newcommand{\strategyext}[3]{#1~\strategy{#2}{#3}}


\newcommand{\featopt}{\strategy{\featkw}{\optkw}}
\newcommand{\statselfopt}{\strategy{\statself}{\optkw}}
\newcommand{\stattotalopt}{\strategy{\stattotal}{\optkw}}

\newcommand{\featcon}{\strategy{\featkw}{\conkw}}
\newcommand{\statselfcon}{\strategy{\statself}{\conkw}}
\newcommand{\stattotalcon}{\strategy{\stattotal}{\conkw}}

\newcommand{\featcostopt}{\strategyext{\featkw}{\costkw}{\optkw}}
\newcommand{\statselfcostopt}{\strategyext{\statself}{\costkw}{\optkw}}
\newcommand{\stattotalcostopt}{\strategyext{\stattotal}{\costkw}{\optkw}}


\newcommand{\featcostcon}{\strategyext{\featkw}{\costkw}{\conkw}}
\newcommand{\statselfcostcon}{\strategyext{\statself}{\costkw}{\conkw}}
\newcommand{\stattotalcostcon}{\strategyext{\stattotal}{\costkw}{\conkw}}

\newcommand{\featconf}{\strategy{\featkw}{\confkw}}
\newcommand{\statselfconf}{\strategy{\statself}{\confkw}}
\newcommand{\stattotalconf}{\strategy{\stattotal}{\confkw}}

\newcommand{\randomopt}{\strategy{\randkw}{\optkw}}
\newcommand{\randomcon}{\strategy{\randkw}{\conkw}}

\newcommand{\toggle}{\togglekw}

%% types

\newcommand{\type}{\ensuremath{t}}

\newcommand{\deep}{\ensuremath{Deep}}

\newcommand{\shallow}{\ensuremath{Shallow}}


%% program, components, strategy

\newcommand{\program}{\ensuremath{P}}


\newcommand{\component}{\ensuremath{c}}

\newcommand{\strategyvar}{\ensuremath{S}}


%% lattice

\newcommand{\latticeL}{\mathcal{L}}

\newcommand{\lattice}[1]{\ensuremath{\latticeL\llbracket#1\rrbracket}}

\newcommand{\standardlattice}{\lattice{\system}{\kmap}}

\newcommand{\conf}{\ensuremath{k}}

\newcommand{\metric}{\ensuremath{\leq_{\latticeL}^{X}}}


\newcommand{\modem}{\ensuremath{M}}
\newcommand{\mode}[1]{\ensuremath{\modem\llbracket#1\rrbracket}}

\newcommand{\orderkw}{\ensuremath{<}} 
\newcommand{\ordered}[2]{\ensuremath{#1 \orderkw #2}}

\newcommand{\orderqkw}{\ensuremath{\leqslant}} 
\newcommand{\orderqed}[2]{\ensuremath{#1 \orderqkw #2}}

%% performance


\newcommand{\slowdownkw}{\ensuremath{slowdown}}
\newcommand{\slowdown}[2]{\ensuremath{\slowdownkw(#1,#2)}}

\newcommand{\takikawa}{\ensuremath{T}}




\begin{document}

\title{How Profilers Can Help Navigate Type Migration}

\author{Ben Greenman}
\orcid{0000-0001-7078-9287}
\affiliation{%
  \institution{PLT @ Brown University}
  \city{Providence}
  \state{Rhode Island}
  \country{USA}
}
\email{benjaminlgreenman@gmail.com}

\author{Matthias Felleisen}
\orcid{0000-0001-6678-1004}
\affiliation{%
  \institution{PLT @ Northeastern University}
  \city{Boston}
  \state{Massachusetts}
  \country{USA}
}
\email{matthias@ccs.neu.edu}

\author{Christos Dimoulas}
\orcid{0000-0002-9338-7034}
\affiliation{%
  \institution{PLT @ Northwestern University}
  \city{Evanston}
  \state{Illinois}
  \country{USA}
}
\email{chrdimo@northwestern.edu}

%\renewcommand{\shortauthors}{...}

%%
%% The abstract is a short summary of the work to be presented in the
%% article.
%% -----------------------------------------------------------------------------

\begin{abstract}
  Sound migratory typing envisions a safe and smooth refactoring of
  untyped code bases to typed ones.  However, the cost of enforcing safety
  with run-time checks is often prohibitively high, thus performance
  regressions are a likely occurrence.  Additional types can often
  recover performance, but choosing the right components to type is difficult
  because of the exponential size of the migratory typing lattice. In
  principal though, migration could be guided by an effective
  interpretation of feedback from profiling tools.  To examine this
  hypothesis, this paper  follows the rational programmer method and
  reports on the results of an experiment on tens of thousands of
  performance-debugging scenarios via seventeen strategies for
  turning profiler output into an actionable next step.  The most
  effective strategy is the use of deep types to eliminate the most
  costly boundaries between typed and untyped components; this strategy
  succeeds in more than 50\% of scenarios if a small number of performance
  degradations is tolerable along the way.  
\end{abstract}


%%
%% The code below is generated by the tool at http://dl.acm.org/ccs.cfm.
%% Please copy and paste the code instead of the example below.

\begin{CCSXML}
<ccs2012>
<concept>
<concept_id>10011007.10011006.10011039.10011311</concept_id>
<concept_desc>Software and its engineering~Semantics</concept_desc>
<concept_significance>500</concept_significance>
</concept>
<concept>
<concept_id>10011007.10011006.10011008.10011024.10011032</concept_id>
<concept_desc>Software and its engineering~Constraints</concept_desc>
<concept_significance>100</concept_significance>
</concept>
<concept>
<concept_id>10011007.10011006.10011008.10011009.10011012</concept_id>
<concept_desc>Software and its engineering~Functional languages</concept_desc>
<concept_significance>100</concept_significance>
</concept>
</ccs2012>

\end{CCSXML}

\ccsdesc[500]{Software and its engineering~Semantics}
\ccsdesc[100]{Software and its engineering~Constraints}
\ccsdesc[100]{Software and its engineering~Functional languages}

\newcommand{\code}[1]{\texttt{#1}}
\newcommand{\stdrkt}{\code{v8.6.0.2 [cs]}}
\newcommand{\bmname}[1]{\textsf{#1}}
\newcommand{\totalnumconfigs}{116,163}
\newcommand{\totalnummeasurements}{1,277,793}
%% 2023-03-22 bg: configs + runs ignoring quadT
\newcommand{\machinename}[1]{\texttt{#1}}
\newcommand{\commitname}[2]{\texttt{#1}}
\newcommand{\gcell}[1]{\cellcolor{green!20}#1}
\newcommand{\wcell}[1]{\cellcolor{black!05}#1}
\newcommand{\ycell}[1]{\cellcolor{yellow!18}#1}
\newcommand{\ocell}[1]{\cellcolor{orange!29}#1}
\newcommand{\rcell}[1]{\cellcolor{red!30}#1}
\newcommand{\tcell}[1]{\cellcolor{black!10}#1}

\keywords{complete monitoring, blame soundness, blame completeness}

\maketitle

%% -----------------------------------------------------------------------------
\def\sec#1#2{\section{#2} \label{sec:#1} \input{#1.tex}}

\sec{intro}       {Type Migration as a Navigation Problem}
\sec{seascape}    {Navigating the Deeps and Shallows via Profiling Tools}
\sec{ideas}       {A Rational Approach to Navigation}
\sec{experiment}  {From a Rational Approach to a Rational Programmer Experiment}
\sec{results}     {What are the Results of the Rational Programmer Experiment}
\sec{discussion}  {What Can Programmers Learn from the Rational Programmer}
\sec{related}     {What Does Prior Research Say About This Problem}
\sec{conclusion}  {Where to Go from Here}
%% -----------------------------------------------------------------------------


\begin{acks}
  Thanks to Cloudlab for hosting our experiments [CITE].

We gratefully acknowledge support from
  \grantsponsor{NSF}{NSF}{https://www.nsf.gov}
 grants
  (?) \href{"https://www.nsf.gov/awardsearch/showAward?AWD_ID=1763922"}{\grantnum{NSF}{CCF 1763922}},
  (?) \href{"https://www.nsf.gov/awardsearch/showAward?AWD_ID=1823244"}{\grantnum{NSF}{CNS 1823244}},
 and
 \href{"https://www.nsf.gov/awardsearch/showAward?AWD_ID=2030859"}{\grantnum{NSF}{CCF 2030859}}
  to the CRA for the \href{https://cifellows2020.org}{CIFellows} project.
\end{acks}

\appendix

\section{FILL Skyline}

\begin{figure}[t]
  \newcommand{\kkrow}[1]{\bmname{#1} \\ \includegraphics[width=0.39\columnwidth]{data/sky/#1-feasible.pdf}}
  \begin{tabular}[t]{ll}
    \begin{tabular}[t]{l}
      \kkrow{morsecode} \\
      \kkrow{forth} \\
      \kkrow{fsm} \\
      \kkrow{fsmoo} \\
      \kkrow{mbta} \\
      \kkrow{zombie} \\
      \kkrow{dungeon} \\
      \kkrow{jpeg} \\
    \end{tabular}
    &
    \begin{tabular}[t]{l}
      \kkrow{lnm} \\
      \kkrow{suffixtree} \\
      \kkrow{kcfa} \\
      \kkrow{snake} \\
      \kkrow{take5} \\
      \kkrow{acquire} \\
      \kkrow{tetris} \\
      \kkrow{synth}
    \end{tabular}
  \end{tabular}

  \caption{FILL skyline}
  \label{f:skyline:bm}
\end{figure}


\section{FILL Head to Head}

\begin{figure}[t]
  \newcommand{\hhrow}[1]{\bmname{#1} \\ \includegraphics[width=0.39\columnwidth]{data/h2h/#1.pdf}}
  \begin{tabular}[t]{ll}
    \begin{tabular}[t]{l}
      \hhrow{morsecode} \\
      \hhrow{forth} \\
      \hhrow{fsm} \\
      \hhrow{fsmoo} \\
      \hhrow{mbta} \\
      \hhrow{zombie} \\
      \hhrow{dungeon} \\
      \hhrow{jpeg} \\
    \end{tabular}
    &
    \begin{tabular}[t]{l}
      \hhrow{lnm} \\
      \hhrow{suffixtree} \\
      \hhrow{kcfa} \\
      \hhrow{snake} \\
      \hhrow{take5} \\
      \hhrow{acquire} \\
      \hhrow{tetris} \\
      \hhrow{synth}
    \end{tabular}
  \end{tabular}

  \caption{FILL Head to head, per benchmark}
  \label{f:h2h:bm}
\end{figure}

\bibliographystyle{ACM-Reference-Format}
\bibliography{bib}

\end{document}
\endinput
