%% -----------------------------------------------------------------------------

Sound migratory typing promised a safe and smooth refactoring path from an
untyped code base to a typed one~\cite{tf-dls-2006, tfffgksst-snapl-2017}. It
realizes the safe part with the compilation of type interfaces to run-time
checks that guarantee type-level integrity of each mixed-typed program configuration.
Unfortunately, these run-time checks impose a large performance
overhead~\cite{gtnffvf-jfp-2019}, making the path anything but smooth. While
this problem is particularly stringent for deep (or natural~\cite{tf-dls-2006,
tf-popl-2008, st-sfp-2006}) run-time checks, it also applies to shallow (or
transient~\cite{vss-popl-2017,vksb-dls-2014,v-thesis-2019}) run-time type
checking~\cite{gm-pepm-2018}.\footnote{\citet{kas-pldi-2019} have demonstrated
that, for a lambda-calculus-sized programming language, it might be possible to
reduce the overhead dramatically. This research is yet to be applied to a large,
in-use language.}

Recently, \citet{g-thesis-2020,g-deep-shallow} presented evidence that deep and
shallow checks come with complementary strengths and weaknesses. Deep checking
imposes a steep cost yet enables type-driven optimizations that off-set
some---and sometimes more than all---of the cost as more type annotations are
added. By contrast, shallow checking is a pay-as-you-go scheme; a developer pays
only for the type annotations added and the worst-case cost seems to be capped
at an order-of-magnitude. Hence, Greenman argues that developers should, in
principle, be able to mix and match deep and shallow checking to get the
best-possible type checking benefits with a tolerable performance penalty.
Greenman has implemented this idea for Typed Racket and initial empirical data
is promising: with a mixture of checks developers can avoid migration paths that
result in unacceptable slowdowns with either deep or shallow checks alone.
However, the mixture of deep and shallow checks makes it even more difficult to
find a smooth migration path. With a purely shallow or purely deep type checking
scheme, developers have to pick from $2^N$ configurations of a program with $N$
components that can be typed or untyped. In a context with both deep and shallow
checking, the size of the configurations space goes to $3^N$.

This situation raises the question 
\begin{quote} \em
 how developers can navigate the migration lattice of a code base while keeping
 the performance penalty from run-time checks at acceptable levels.
\end{quote}  
Obviously, such a situation calls for the use of profiling tools. But, this
response merely refines the above question in two ways, namely, 
\begin{itemize} \em

\item how a developer should react to feedback from various profiling tools; and 

\item how well feedback helps find a path from a bad configuration to one with
 tolerable performance.

\end{itemize}   

These questions call for an empirical investigation, though using human
developers for a large number of performance problems is too costly and too
error-prone. Instead, this paper reports on the results of a \emph{rational
programmer} experiment~\cite{,lgfd-icfp-2021}. The method employs algorithmic
abstractions, dubbed rational programmers, that reify a falsifiable hypothesis
about various ways of using profiling tools and interpreting their feedback.
These algorithmic abstractions are then applied to a diverse population of
problematic performance scenarios. A rational-programmer experiment simulates the
satisficing behavior of developers in a large number of performance-debugging
situations and thus enables a systematic comparison of different rational
programmers.

In short, this paper makes two contributions:
\begin{itemize}

\item At the object level, the results of the rational programmer experiment
 provide guidance to developers about how to best use the feedback from
 profilers during type migration.

\item At the meta level, the application of the rational programmer method to
 the performance problems of type migration provides one more piece of evidence
 about its usefulness. 
    
\end{itemize}    
The remainder of the paper is organized as follows.  Section~\ref{sec:seascape}
uses examples to demonstrate how deep and shallow checks can cause performance
bottlenecks during type migration, and how profiler feedback can help developers
indentify and mitigate them.  Section~\ref{sec:ideas} explains how the rational
programmer method makes it possible to systematically evaluate the effectiveness
of different strategies for interpreting profiler
feedback. Section~\ref{sec:experiment} lays out the details of how these ideas
translate to a large-scale quantitative experiment.  Section~\ref{sec:results}
presents the results of the experiment, and section~\ref{sec:discussion}
extracts lessons for developers and researchers.  Section~\ref{sec:related}
places this work in the context of prior research, and
section~\ref{sec:conclusion} puts this work in perspective with respect to
future research.
