\newcommand{\gtpurl}{\url{https://docs.racket-lang.org/gtp-benchmarks/index.html}}
%% -----------------------------------------------------------------------------

Gradual type migration is an open and challenging problem~\cite{rch:in-out-infer-gt, km:ts-type-evo,
mp:gt-decidable, ccew:gt-migrate, gc:gt-infer,
cagg-solver-based-migration,clps-popl-2020,js-infer,ruby-static-infer,unif-infer,
msi:gt-infer-hm,dyn-infer-ruby,profile-guided-typing,jstrace,gen-ts-decl,
lambdanet,nl2ptype,learn-types-big-data,ml-ts, ccw-icfp-2018}.
For any untyped component, a migrating developer has to choose
practical type annotations from among an often-infinite number of theoretical
ones. But, to make a rational programmer experiment computationally feasible, it
is necessary to avoid this dimension.

Fortunately, the construction of the corpus of scenarios from a carefully
selected set of suitable seed programs can solve the problem.
The established GTP
benchmarks~\cite{gtnffvf-jfp-2019,g-rep-2023} are
representative of the programming styles in the Racket world, and they
come with well-chosen type annotations for all their components.  Hence,
the migration lattices can be pre-constructed for all benchmark programs.
It is thus possible to apply a strategy to any performance-debugging
scenario (a program with intolerable performance) in this lattice and
use the strategy's recommendations
to chart a path through the program's migration lattice.

Intuitively, a strategy \strategyvar{} attempts to convert a program $\program{}_0$
into an improved program $\program{}_n$ in  a step-wise manner.
Each intermediate point $\program{}_i$ from $\program{}_0$ to
$\program{}_n$ is the result of applying the \strategyvar{}
to the current program.
In essence,
$\strategyvar{}$ constructs a \emph{migration path}, a sequence of
programs ${\program{}_0}, \ldots, {\program{}_n}$ from a migration
lattice. If \strategyvar{} cannot make a recommendation at any point along
this path, migration halts.
The following definitions formalize these points.

\paragraph{The Migration Lattice.}  All programs $\program{}_i$ are nodes in the
\emph{migration lattice} \lattice{\program{}_t} where ${\program{}_t}$, is like
${\program{}_i}$ but all its components have types (either deep or
shallow).\footnote{Although there are several possible choices for
$\program_t$, each denotes a unique lattice. By contrast, a lattice
based an untyped program (\lattice{\program{}_u}) is ambiguous without a
pre-determined set of types.}
In other words, a component in ${\program{}_i}$ may have no types 
or toggled types compared to ${\program{}_t}$.
The bottom element of \lattice{\program{}_t} is ${\program{}_u}$, the untyped
program.
The $3^N$ nodes of \lattice{\program{}_t} are ordered:
$\ordered{\program{}_i}{\program{}_j}$ iff
the untyped components in ${\program{}_i}$ are a subset
of those in ${\program{}_j}$.
Hence the lattice is organized in \emph{levels} of incomparable
configurations.
Every configuration in the same level has the same set of untyped components
but a distinct combination of deep and shallow types for the typed ones.
The notation $\orderqed{\program{}_i}{\program{}_j}$ denotes that either
$\ordered{\program{}_i}{\program{}_j}$ or $\program{}_i$ and $\program{}_j$
are at the same level.

A migration path corresponds to a collection of configurations $\program{}_i$,
$0 \leq i < n$, such that $\orderqed{\program{}_i}{\program{}_{i+1}}$.
This statement is the formal equivalent to the
description from the preceding section that strategies either add types to a
single previously untyped component or toggle the type enforcement regime of existing
typed components.
(No strategy, including the agnostic ones, modifies a boundary where both sides
are untyped.)
In other words, a migration path is a weakly ascending chain in
\lattice{\program{}_t}.


\paragraph{Performance-debugging scenarios and success criteria.} Completing the
formal description of the experiment demands answers to two more questions.
The first concerns the selection of the starting points for the strategy-driven
migrations, i.e., the \emph{performance-debugging scenarios}.
Which configurations $\program_i$ quality as slow?
Since type checks are the source of performance overhead, the appropriate
way to measure costs is by comparing $\program_i$ to the untyped configuration
$\program_u$:

\begin{quote} \em

Given a migration lattice, a \emph{performance-debugging scenario} is a
  configuration $\program$ such that \slowdown{\program}{\program{}_u} > \takikawa{}.
  \begin{itemize} \item[] \begin{itemize}
    \item
      \slowdown{\program}{\program{}_u} is the ratio of the performance of
      \program{} over that of $\program{}_u$
    \item
      $\takikawa{}$ signifies the maximum acceptable performance degradation
  \end{itemize} \end{itemize}
\end{quote}

The second question is about differentiating successful from failing migrations.
Strictly speaking, performance should
always improve, otherwise the developer may not wish to invest any more effort
into migration.
In the worst case, performance might stay the same for a few
migration steps before it becomes acceptable:

\begin{quote} \em

A migration path ${\program{}_0} \ldots {\program{}_n}$ in a lattice \lattice{\program{}_t}
is \emph{strictly successful}
iff
\begin{enumerate}
  \item $\program{}_0$ is a performance-debugging scenario,
  \item $\slowdown{\program{}_n}{\program{}_u} \leq \takikawa{}$, and 
  \item for all $0 \leq i < n$, $\slowdown{\program{}_{i+1}}{\program{}_{i}} \leq 1$.
 \end{enumerate} 
\end{quote}
To achieve strict success, a strategy must monotonically improve performance.

An alternative to strict success is to tolerate occasional setbacks.
Accepting that a migration path may come with $k$ setbacks where performance
gets worse, a $k$-loose success relaxes the requirement for monotonicity $k$
times:

\begin{quote} \em

A migration path ${\program{}_0} \ldots {\program{}_n}$ in a
lattice \lattice{\program{}_t} is \emph{$k$-loosely successful}
iff 
\begin{enumerate}
  \item  $\program{}_0$ is a performance-debugging scenario,
  \item $\slowdown{\program{}_n}{\program{}_u} \leq \takikawa{}$  
  \item
      for all $0 \leq i < n$ with at most $k$ exceptions,
      $\slowdown{\program{}_{i+1}}{\program{}_{i}} \leq 1$
      \subitem equivalently: $k \geq |\{ \slowdown{\program{}_{i+1}}{\program{}_{i}} > 1 \mid 0 \leq i < n\}|$
  \end{enumerate} 
\end{quote}
The construction of a $k$-loose successful migration path allows a strategy to
temporarily degrade performance. The constant $k$ is an
upper bound on the number of missteps.

A patient developer may tolerate an unlimited number of setbacks:

\begin{quote} \em
A migration path ${\program{}_0} \ldots {\program{}_n}$ is $N$-loosely successful if 
  \begin{enumerate}
  \item  $\program{}_0$ is a performance-debugging scenario,
  \item $\slowdown{\program{}_n}{\program{}_u} \leq \takikawa{}$  
  \end{enumerate}
\end{quote}


