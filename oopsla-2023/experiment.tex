%% -----------------------------------------------------------------------------

Turning the sketch from section~\ref{sec:ideas} into a large-scale
automated experiment rests on implementing the reactions of the rational
programmer to profiler feedback. In essence, this requires constructing an
algorithm for each mode of the rational programmer that implements the
strategy-specific piece of the mode. All these algorithms share a  common
task: they have to modify a program as dictated by the
strategy of a mode.  In particular, the modification may ask the addition
of types to an untyped component, i.e., the gradual type migration of the
component.  However, gradual type migration is still an open and challenging
problem for realistic components~\cite{rch:in-out-infer-gt,
mp:gt-decidable, ccew:gt-migrate, msi:gt-infer-hm, gc:gt-infer,
cagg-solver-based-migration, km:ts-type-evo}.
Fortunately~\citep{lgfd-icfp-2021} show a way forward. If the program
corpus of the experiment is based on a set of fully-typed programs then
the types of each component are known in advance, and moreover, their
migration  lattices~\citep{tfgnvf-popl-2016} can be pre-constructed.
Hence, a rational programmer mode starts from a program configuration
that exhibits intolerable performance, that is a performance-debugging
scenario, and uses the known types of the components to chart a path
through the program's migration lattice guided by its profiling strategy.

As section~\ref{sec:ideas} describes, each profiling strategy interprets
profiler feedback differently, and suggests a corresponding program
modification. Hence, in addition to the common for all modes part that
modifies a program, an automated experiment demands the implementation of
the strategies it aims to compare. With the mode implementations in hand,
the experiment runs all of them on the same large number of scenarios and
compares their successes. The rest of the section provides precise
definitions for all these elements of the experiment.
Section~\ref{subsex:lattice}) defines the migration lattice of a program
and characterizes performance-debugging scenarios.
Section~\ref{subsec:strategies} presents the strategies that induce the
modes of the rational programmer for the experiment together with the
criteria for their success and failure. Finally,
section~\ref{subsec:questions} leverages these definitions to formalize
the experimental questions and the experimental procedure to answer
them.

\subsection{Migration Lattices and Performance-Debugging Scenarios}
\label{subsec:lattice}

\subsection{The Rational-Programmer Strategies}
\label{subsec:strategies}

\subsection{The Experimental Questions}
\label{subsec:questions}

