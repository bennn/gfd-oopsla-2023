%% -----------------------------------------------------------------------------

Turning the sketch from section~\ref{sec:ideas} into a large-scale
automated experiment rests on implementing the reactions of the rational
programmer to profiler feedback. In essence, this requires constructing an
algorithm for each mode of the rational programmer that implements the
strategy-specific piece of the mode. All these algorithms share a  common
task: they have to modify a program as dictated by the
strategy of a mode.  In particular, the modification may ask the addition
of types to an untyped component, i.e., the gradual type migration of the
component.  However, gradual type migration is still an open and challenging
problem for realistic components~\cite{rch:in-out-infer-gt,
mp:gt-decidable, ccew:gt-migrate, msi:gt-infer-hm, gc:gt-infer,
cagg-solver-based-migration, km:ts-type-evo}.
Fortunately~\citep{lgfd-icfp-2021} show a way forward. If the program
corpus of the experiment is based on a set of fully-typed programs then
the types of each component are known in advance, and moreover, their
migration  lattices~\citep{tfgnvf-popl-2016} can be pre-constructed.
Hence, a rational programmer mode starts from a program configuration
that exhibits intolerable performance, that is a performance-debugging
scenario, and uses the known types of the components to chart a path
through the program's migration lattice guided by its profiling strategy.

As section~\ref{sec:ideas} describes, each profiling strategy interprets
profiler feedback differently, and suggests a corresponding program
modification. Hence, in addition to the common for all modes part that
modifies a program, an automated experiment demands the implementation
of the strategies it aims to compare. With the mode implementations in
hand, the experiment runs all of them on the same large number of
scenarios and compares their successes. The rest of the section provides
precise definitions for all these elements of the experiment.
Section~\ref{subsec:strategies} presents the profiling strategies that induce the
modes of the rational programmer for the experiment.
Section~\ref{subsec:lattice} defines the migration lattice of a program,
which the modes navigate, characterizes the performance-debugging
scenarios that serve as starting navigation points, and lays out  the
criteria for a successful or failed navigation. Finally,
section~\ref{subsec:questions} leverages these definitions to formalize
the experimental questions and the experimental procedure that answers
them.

\subsection{The Profiling Strategies}
\label{subsec:strategies}

From the perspective of the experiment, every program \program{} is a
collection of interacting components $\component$. Specifically, some
components have deep types, some  have shallow, and some  have no types at
all. Independently of their types, a component $\component{}_1$, may import
another component $\component{}_2$  which establishes a
\emph{boundary} between them through which they exchange values at
run-time. As section~\ref{sec:seascape} describes, depending on the kind of types at
the two sides of the boundary, a value exchange can result in checks,
which may affect negatively the performance of \program{}. 

 Since boundaries are the source of performance degradation, a profiling
 strategy aims to change the types of components in order to eliminate a
 boundary in a program, hopefully the most costly one. In formal terms, a
 profiling strategy is a function that consumes a program \program{} and
 returns a set of pairs $(\component{}, \type{})$ where \type{} is either
 \deep{} or \shallow{} for deep and shallow respectively. Each pair
 prescribes a modification for \program{}. For instance, if a strategy
 returns the singleton set with the pair $(a,\deep)$, then the strategy
 directs its rational-programmer mode to equip component $a$ with deep
 types, or to toggle to deep its existing types, if $a$ is already
 typed. The strategy returns the empty set when it cannnot determine a
 modification.

 Overall, the experiment compares rational-programmer modes
 derived from 9 different basic strategies. Figure~\ref{f:bstrategies}
 lists them along with their descriptions.

 At a first level, the strategies
 differ in whether they produce their result by analyzing the feedback of
 a \featkw{} or a \statkw{} profiler. Those that use the first kind of
 profiler identify directly the most costly boundary in 
 \program{}, and produce modifications of the components at either side.
 Those that use the second cannot directly identify a boundary.  Instead
 they detect the most costly \component{} in terms of either \selfkw{} or
 \totalkw{} time.  Then, they pick  one of the boundaries that \component{} is
 involved in, which they calculate based on the components that
 \component{} imports or that import \component{}. Only after figuring out
 their target boundary, \statkw{} strategies select how to modify the
 components it involves.

 At a second level, strategies also differ in how they modify the two sides of
 their target boundary. Strategies that are \optkw{} turn the types at
 either side of the boundary to deep. After all, when two components with
 deep types interact, no checks take place. Hence, such a modification
 eliminates the cost of the boundary entirely. However, there is a risk or
 a ripple effect: the new deep types of the two components may increase
 the cost of other boundaries that the two components are involved in. In
 contrast to strategies that are \optkw{}, those that are \conkw{} turn
 the types at either side of the target boundary to shallow. The rationale
 behind this choice is that if both sides of a boundary have
 shallow types, the interactions across the boundary cost less than if
 only one is deep. At the same time, unlike the  \optkw{}
 strategies,  there is no risk of a ripple effect. 

 Besides the basic strategies, the experiment investigates 9 composite
 strategies (fig. ~\ref{f:cstrategies}). The \costkw{} strategies rank the cost of boundaries 
 not just in terms of performance, but also in terms of labor. 
 Hence, they give priority to those
 boundaries that involve typed components as modifying these components
 boils down to toggling their types, which is cheap. The \confkw{}
 strategies
 decide whether to be \optkw{} or \conkw{} best on how many components of
 a program are typed. When most components are untyped, the risk of a ripple effect
 outweighs the benefits of 
 the \optkw{} strategy. Therefore strategies favor a \conkw{} approach for
 sparsely typed programs, and an \optkw{} strategy for densely typed ones.

 Finally, the experiment includes 3 strategies that are \agnostickw{}
 (fig. ~\ref{f:astrategies}). Those
 play the role of baselines for rejecting the null-hypothesis: if 
 profiling strategies are more successful than the
 \agnostickw{} ones, then profiler feedback plays a meaningful role in
 guiding type migration. There are two kinds of  \agnostickw{} strategies. 
 First, the \randkw{} strategies pick a boundary without consulting a profiler
 and modify it either in an \optkw{} or \conkw{} manner. Second, the
 \togglekw{} strategy~\cite{g-dsgt} simply toggles all the typed components of a program
 from deep to shallow and vice versa. 

 %% -----------------------------------------------------------------------------
\begin{figure}[htb]

  \newcommand{\desc}[1]{\parbox[t]{26em}{#1}}

  \def\desca{\desc{Uses the feature-specific profiler to identify the most
  expensive boundary. Then it equips both of its sides with deep types.}}

  \def\descb{\desc{Uses the feature-specific profiler to identify the most
  expensive boundary. Then it equips both of its sides with shallow types.}}

  \def\descc{\desc{Uses the statistical profiler to identify the component
  $\component{}_1$ with the highest self time that shares a boundary with
  a component $\component{}_2$ s.t.  $\component{}_2$ has the  the highest
  self time among all components that share a boundary with
  $\component{}_1$, and $\component{}_2$ and $\component{}_1$ do not have
  both deep or shallow types. Then it equips both sides of the boundary
  with deep types.}}

  \def\descd{\desc{Uses the statistical profiler to identify the component
  $\component{}_1$ with the highest total time that shares a boundary with
  a component $\component{}_2$ s.t.  $\component{}_2$ has the  the highest
  total time among all components that share a boundary with
  $\component{}_1$, and $\component{}_2$ and $\component{}_1$ do not have
  both deep or shallow types. Then it equips both sides of the boundary
  with deep types.}}


  \def\desce{\desc{Uses the statistical profiler to identify the component
  $\component{}_1$ with the highest self time that shares a boundary with
  a component $\component{}_2$ s.t.  $\component{}_2$ has the  the highest
  self time among all components that share a boundary with
  $\component{}_1$, and $\component{}_2$ and $\component{}_1$ do not have
  both deep or shallow types. Then it equips both sides of the boundary
  with shallow types.}}

  \def\descf{\desc{Uses the statistical profiler to identify the component
  $\component{}_1$ with the highest total time that shares a boundary with
  a component $\component{}_2$ s.t.  $\component{}_2$ has the  the highest
  total time among all components that share a boundary with
  $\component{}_1$, and $\component{}_2$ and $\component{}_1$ do not have
  both deep or shallow types. Then it equips both sides of the boundary
  with shallow types.}}




   \begin{tabular}{r|l}
    {\bf Strategy Name} & {\bf Description} \\ \hline
    %%%%%%%%%%%%%%%%%%%%%%%%%%%%%%%%%%%%%%%%%%%%%%%%%%%%%%%%%%%%%%%%
     \featopt{}          &   \desca          \\ \hline
    \featcon{}          &   \descb           \\ \hline

    \statselfopt{}      &   \descc           \\ \hline
    \stattotalopt{}     &   \descd           \\ \hline
    \statselfcon{}      &   \desce           \\ \hline 
    \stattotalcon{}     &   \descf           
    \end{tabular}

 
  \caption{The Basic Strategies of the Experiment and their Descriptions}
  \label{f:bstrategies}
\end{figure}
%% -----------------------------------------------------------------------------

 %% -----------------------------------------------------------------------------
\begin{figure}[htb]
 
  \newcommand{\desc}[1]{\parbox[t]{21em}{#1}}

 
  \def\desca{\desc{Splits the boundaries in the given program to those
  between typed components and the rest.
  Uses the feature-specific profiler to identify the most
  expensive boundary from the first group. 
  Then it equips both of its sides with deep types.
  If the first gorup is empty, it uses the second group.}}

   \def\descb{\desc{Splits the boundaries in the given program to those
  between typed components and the rest.
  Uses the feature-specific profiler to identify the most
  expensive boundary from the first group. 
  Then it equips both of its sides with shallow types.
  If the first gorup is empty, it uses the second group.}}



 \begin{tabular}{r|l}
    {\bf Strategy Name} & {\bf Description} \\\hline
    %%%%%%%%%%%%%%%%%%%%%%%%%%%%%%%%%%%%%%%%%%%%%%%%%%%%%%%%%%%%%%%%
    \featcostopt{}      &   \desca           \\\hline
    \featcostcon{}      &   \descb           \\\hline
  
    \statselfcostopt{}  &                    \\\hline
    \stattotalcostopt{} &                    \\\hline
   \statselfcostcon{}   &                    \\\hline
    \stattotalcostcon{} &                    \\\hline

    \featconf{}         &                    \\\hline
    \statselfconf{}     &                    \\\hline
    \stattotalconf{}    &                    \\
    \end{tabular}

 
  \caption{The Composite Strategies of the Experiment and their Descriptions}
  \label{f:cstrategies}
\end{figure}

 %% -----------------------------------------------------------------------------
\begin{figure}[htb]

  \newcommand{\desc}[1]{\parbox[t]{26em}{#1}}

  \def\desca{\desc{Uses the feature-specific profiler to identify the most
  expensive boundary. Then it equips both of its sides with deep types.}}


 \begin{tabular}{r|l}
    {\bf Strategy Name} & {\bf Description} \\\hline
    %%%%%%%%%%%%%%%%%%%%%%%%%%%%%%%%%%%%%%%%%%%%%%%%%%%%%%%%%%%%%%%%
    \randomopt{}        &                    \\\hline
    \randomcon{}        &                    \\\hline
    \toggle{}           &                    
     \end{tabular}

 
  \caption{The Profiler-Agnostic Strategies of the Experiment and their Descriptions}
  \label{f:astrategies}
\end{figure}





\subsection{Migration Lattices and their Navigation}
\label{subsec:lattice}
Given a strategy \strategyvar{}, the corresponding mode of the rational, 
$\mode{\strategyvar{}}$, is a function from a program $\program{}_0$ to a
program $\program{}_n$. $\mode{\strategyvar{}}$ produces  $\program{}_n$
by iteratively calling \strategyvar{} on its input and applying its
recommendations to obtain a new potential input for \strategyvar{}. 
In other words, intentionally,  $\mode{\strategyvar{}}$ constructs a
\emph{migration path}, that is a sequence of programs 
${\program{}_0} ... {\program{}_n}$ that differ in their types but share
the same code otherwise.

Put differently, all programs $\program{}_i$ are nodes on the
\emph{migration lattice} \lattice{\program{}_u} where ${\program{}_u}$, is
like ${\program{}_i}$ but all its components are untyped. The bottom
element of \lattice{\program{}_u} is ${\program{}_u}$ itself, and its top
elements are ${\program{}_u}$'s fully typed versions $\program{}_t$ (with some
combination of Deep and Shallow types). In between these extreme points
are all the reamining \emph{configurations} of $\program{}_u$. The $3^N$
configurations of  \lattice{\program{}_u} are ordered:
$\ordered{\program{}_i}{\program{}_j}$ iff $\program{}_j$ has at least one
component that is untyped in $\program{}_j$. Hence, the lattice is
organized in \emph{levels} of incomperable configurations that have the same
untyped components, but different choices for which of their typed
components have Deep or Shallow types.
$\orderqed{\program{}_i}{\program{}_j}$ denotes that either $\program{}_i$
and $\program{}_j$ are at the same level of \lattice{\program{}_u} or
$\program{}_i$ is at some level below $\program{}_j$.

Given, these definitions a migration path corresponds to a collection of
configurations  $\program{}_i$, $0 \leq  i \leq n$, such that
$\orderqed{\program{}_i}{\program{}_i+1}$. In fact, either  $\program{}_i$
and $\program{}_{i+1}$ are at the same level or they are at two
``neighboring'' ones.  Essentially, since that goal of all strategies
from~\ref{subsec:strategies} is to eliminate  boundaries that impose
run-time type checks, they either add types to a single previously
untyped component or toggle the types of existing typed
components.\footnote{None of the strategies modifies a boundary where both
side are untyped. Such boundaries are invisible to the feature-specific
profiler, and the strategies that use the statistical one explictely
filter them out. Same for the profiler-agnostic ones.} 
In other words, a migration path is a weakly ascending
chain in \lattice{\program{}_u}.

This latter observation is key for the implementation of the
rational-programmer experiment. If, for each  $\program{}_u$ in the
experiment, one of its fully type versions $\program{}_t$ is also
available, then the construction of \lattice{\program{}_u} is
automatic, and as a result, so are the modifications that the strategies 
produce. Therefore, given implementations of the strategies and a
collection of seed programs with both untyped and fully typed versions,
the rational-programmer experiment reduces to a fully automated push-button 
process that asks the all modes of the rational programmer to
construct a migration path starting at the same configurations of   
\lattice{\program{}_u}, and compares how succesful these path are. 

The discussion reveals the two remaining unaswered questions for a complete
operational description of the rational-programmer experiment. The first
question is the
selection of the starting points given to the modes of the rational
programmer, i.e., the \emph{performance-debugging scenarios}. 
The answer is simple: 
every configuration $\program$ that exhibits performance degradation compared to 
$\program{}_u$ is a performance-debugging scenarios. In formall terms: 

\begin{quote}
\it Given a migration lattice \lattice{\program{}_u}, a configuration $\program$
  in \lattice{\program{}_u} is a \emph{performance-debugging scenario}
  iff \slowdown{\program}{\program{}_u} > 1, where \slowdownkw{} is
  calculated the standard way as the ratio of \program{} over $\program{}_u$.  
\end{quote}

The second remaining question is what consitutes a succesful migration.
For that the experiment employs two different answers. First, 
\begin{quote}
\it A migration path ${\program{}_0}
  ... {\program{}_n}$ in a lattice \lattice{\program{}_u}
  is \emph{strictly successful} iff $\program{}_0$ 
  is a performance-debugging scenario, forall $0 < i < n$, 
  $\slowdown{P_i}{P_i+1} \leq 1$, and $\slowdown{P_n}{P_u} \leq \takikawa{}$.
\end{quote}









\subsection{The Experimental Questions}
\label{subsec:questions}
%% -----------------------------------------------------------------------------

The definitions of successful migration paths enables rephrasing the
questions from~\ref{sec:intro} in terms of strategies:
\begin{itemize}
\item[$Q_X$] Is strategy $X$ successful in the context of the elimination of
  performance bottlenecks via migration?

\item[$Q_{X/Y}$] Is strategy $X$ more successful than strategy $Y$ in this
  context?
\end{itemize}

Answering $Q_X$ boils down to determining the success and failures of
$\mode{X}$, the rational-programmer mode of $X$, for all
performance-debugging scenarios. If, for a large number of scenarios,
$\mode{X}$ charts migration paths that are strictly successful, the answer
is positive. Essentially, the large number of scenarios is evidence that
when a rational programmer reacts to profiler feedback following $X$, it is
likely to benefit  while tuning the performance of 
mixed-typed code. Notably, the above description uses the strict notion of
success, which sets a high standard of success. Hence, the rational
programmer not only manages to tune performance at a tolerable level but
each suggestion of its strategy brings the rational programmer closer to
its target. Swapping the notion of strict success for $k$-loose success
relaxes this high standard, and offers answers to $Q_X$ when allowing for
some bounded flexibility in how well the intermediate suggestions of
$X$ help the rational programmer.   

While an answer to $Q_X$ constitutes an evaluation of a strategy $X$ for interpreting
profiler feedback in absolute terms, an answer to $Q_{X/Y}$ reveals the
relative value of $X$ versus another strategy $Y$. Answering this second
question asks whether the proportion of scenarios where $\mode{X}$ charts a
successful path but  $\mode{Y}$ does not is higher that the proportion of
scenarios where  $\mode{Y}$ charts a successful path but $\mode{X}$ does not.
 Imporantly, when $\mode{Y}$ is the $randomkw{}$ strategy, if
the answer to $Q_X$ is positive, then it invalidates the null-hypothesis,
i.e., that the success of $X$ is because of sheer luck.
Of course, the answer may not be clear cut as $X$ and $Y$ may
perform equally well in most scenarios, or may have complementary success 
records. Again, relaxing the notion of success by different factors $k$
may help distinguish  $X$ and $Y$ based on the quality of the feedback
they produce.


In sum, the process for answering $Q_X$ and $Q_{X/Y}$ rests on the
following experimental plan:
\begin{enumerate}
\item Create a large and diverse corpus of performance-debugging scenarios;
\item Calculate the migration paths for each mode of the rational
  programmer for each scenario;
\item Compute and compare the successes and failures of the modes.  
\end{enumerate}



