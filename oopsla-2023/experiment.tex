%% -----------------------------------------------------------------------------

Turning the sketch from section~\ref{sec:ideas} into a large-scale
automated experiment rests on implementing the reactions of the rational
programmer to profiler feedback. In essence, this requires constructing an
algorithm for each mode of the rational programmer that implements the
strategy-specific piece of the mode. All these algorithms share a  common
task: they have to modify a program as dictated by the
strategy of a mode.  In particular, the modification may ask the addition
of types to an untyped component, i.e., the gradual type migration of the
component.  However, gradual type migration is still an open and challenging
problem for realistic components~\cite{rch:in-out-infer-gt,
mp:gt-decidable, ccew:gt-migrate, msi:gt-infer-hm, gc:gt-infer,
cagg-solver-based-migration, km:ts-type-evo}.
Fortunately~\citep{lgfd-icfp-2021} show a way forward. If the program
corpus of the experiment is based on a set of fully-typed programs then
the types of each component are known in advance, and moreover, their
migration  lattices~\citep{tfgnvf-popl-2016} can be pre-constructed.
Hence, a rational programmer mode starts from a program configuration
that exhibits intolerable performance, that is a performance-debugging
scenario, and uses the known types of the components to chart a path
through the program's migration lattice guided by its profiling strategy.

As section~\ref{sec:ideas} describes, each profiling strategy interprets
profiler feedback differently, and suggests a corresponding program
modification. Hence, in addition to the common for all modes part that
modifies a program, an automated experiment demands the implementation
of the strategies it aims to compare. With the mode implementations in
hand, the experiment runs all of them on the same large number of
scenarios and compares their successes. The rest of the section provides
precise definitions for all these elements of the experiment.
Section~\ref{subsec:strategies} presents the profiling strategies that induce the
modes of the rational programmer for the experiment.
Section~\ref{subsec:lattice} defines the migration lattice of a program,
which the modes navigate, characterizes the performance-debugging
scenarios that serve as starting navigation points, and lays out  the
criteria for a successful or failed navigation. Finally,
section~\ref{subsec:questions} leverages these definitions to formalize
the experimental questions and the experimental procedure that answers
them.

\subsection{The Profiling Strategies}
\label{subsec:strategies}
%% -----------------------------------------------------------------------------
Every program \program{} is a collection of interacting components $\component$.
Some components enforce types with deep checks, some with shallow ones, and some
are untyped. Independently of their types, a component $\component{}_1$, may
import another component $\component{}_2$, which establishes a \emph{boundary}
between them, across which they exchange values at run time. Depending on the
kind of types at the two sides of the boundary, a value exchange can trigger
run-time checks, which may degrade the performance.

A profiling strategy should thus aim to eliminate the most costly boundaries in
a program. In formal terms, a profiling strategy is a function that consumes a
program \program{} and, after determining its profile, returns a set of pairs
$(\component{}, \type{})$. Here \type{} is either \deep{} or \shallow{}. Each
such pair prescribes a modification of \program{}. For instance, if a strategy
returns the singleton set with the pair $(c, \deep)$, then the strategy directs
its rational-programmer mode to equip component $c$ with types (if necessary)
and set its enforcement strategy to deep; if $c$ is typed, it just requests
toggling from shallow to deep.  If a strategy's result is the empty set, it
cannot figure out how to proceed.

%% -----------------------------------------------------------------------------

\def\with{with {\em total\/} in place of {\em self\/}}

\begin{figure}[htb]

  \newcommand{\desc}[1]{\parbox[t]{24em}{#1\\[-2mm]}}
  
  \def\desca{\desc{Uses the feature-specific profiler to identify the most
    expensive boundary in the given program. Then it directs the mode to equip
    both sides of the target boundary with deep types.}}

  \def\descb{\desc{Like \featopt{} but with shallow types for both
    sides of the target boundary.}}

  \def\descc{\desc{Uses the statistical profiler to identify the component
    $\component{}_1$ with the highest self time in the given program that has a
    boundary with at least one component $\component{}_2$ that has stricter
    types than $\component{}_1$.  Then it picks the component $\component{}_2$
    that has the highest self time, and directs the mode to equip both
    $\component{}_1$ and $\component{}_2$ with deep types.}}

  \def\descd{\desc{Like \statselfopt{} \with}}

  \def\desce{\desc{Like \statselfopt{}, with shallow types for $\component{}_1$, $\component{}_2$}}

  \def\descf{\desc{Like \statselfcon{} with {\em total\/} in place of {\em self\/}}}

 \begin{tabular}{l l l}
    {\bf Profiler} & {\bf Strategy} & {\bf Description} \\ \hline
    %%%%%%%%%%%%%%%%%%%%%%%%%%%%%%%%%%%%%%%%%%%%%%%%%%%%%%%%%%%%%%%%
    \multirow[b]{2}[+5]{*}{{\em feature-specific\/}} & 
        \optkw{}          &   \desca          \\ \relax
     &	\conkw{}          &   \descb          \\ \hline

    \multirow[b]{2}[+7]{*}{{\em statistically\/} ({\em self\/})} &
	\optkw{}      &   \descc          \\ \relax
     &	\conkw{}      &   \desce          \\ \hline

    \multirow{2}{*}{{\em statistically\/} ({\em total\/})} & 
	\conkw{}     &   \descf           \\ \relax
     &	\optkw{}     &   \descd           % \\ \relax
 \end{tabular}

 
  \caption{The Basic Strategies of the Experiment and their Descriptions}
  \label{f:bstrategies}
\end{figure}
%% -----------------------------------------------------------------------------

\paragraph{Basic strategies.}  Figure~\ref{f:bstrategies} describes six basic
 strategies that rational programmers may use. 

At the first level, the basic strategies differ in whether they employ the
 \featkw{} or the \statkw{} profiler to profile the given program.  Those that
 use the first kind of profiler can directly identify the most costly boundary
 in \program{} and suggest a modification of either of the two components.
 Those that use the second cannot directly identify a boundary.  Instead they
 focus on the most costly component $\component{}_1$ in terms of either
 \selfkw{} or \totalkw{} time.  From there, they pick one of the boundaries
 between $\component{}_1$ and any of the components that it depends on or that
 depend on it. The goal is to find a boundary between $\component{}_1$ and a
 component $\component{}_2$ where $\component{}_2$ has \emph{stricter} types
 than $\component{}_1$, because in this case, the interactions across the
 boundary are likely to generate costly run-time type checks. Here, deep is
 stricter than shallow, and shallow is stricter than untyped. If the strategy
 cannot identify such a boundary, it moves on to the next costly component
 (again in terms of either \selfkw{} or \totalkw{} time). Once it eventually
 identifies a target boundary, a \statkw{} strategy decides how to migrate the
 two components.

At the second level, basic strategies differ in how they migrate the two sides
 of their target boundary. Strategies that are \optkw{} turn the types at either
 side of the boundary to deep. After all, when two components with deep types
 interact, no checks take place and, better still, deep enforcement may even
 enable type-driven compiler optimizations for the two components.  Hence, such
 a migration may eliminate the cost of the boundary entirely.  Experience shows,
 however, that switching to deep type enforcement for a component may cause a
 ripple effect. The new deeply enforced types may increase the cost of other
 boundaries between (one of) the two components and a third one. By contrast,
 \conkw{} strategies choose shallow enforcement of types for both sides of the
 target boundary. The rationale behind this choice is that, if both sides of a
 boundary have shallow types, the interactions across the boundary cost less
 than if only one is deep and, at the same time, unlike with \optkw{}
 strategies, there is no risk of a ripple effect.

%% -----------------------------------------------------------------------------
\begin{figure}[b]
 
  \newcommand{\desc}[1]{\parbox[t]{19.5em}{#1\\[-2mm]}}
 
  \def\desca{\desc{Splits the boundaries in the given program to those between
    typed components and the rest.  Delegates to \featopt{} to produce a
    modification for the given program, but rank boundaries in the first group
    higher than those in the second group while determining the most expensive
    boundary.}}

  \def\descb{\desc{Like \featcostopt{} but it delegates to \featcon{}.}}

  \def\descc{\desc{Separates the typed components that have boundaries with
    other typed components from the rest of the components in the given
    program. Delegates to \statselfopt{} to produce a modification for the given
    program, but rank boundaries between components in the first group higher
    than the rest while determining the most expensive boundary.}}

   \def\descd{\desc{Like \statselfcostopt{} but it delegates to \stattotalopt{}.}}

   \def\desce{\desc{Like \statselfcostopt{} but it delegates to \statselfcon{}.}}

   \def\descf{\desc{Like \statselfcostopt{} but it delegates to \stattotalcon{}.}}

   \def\descg{\desc{If the number of typed components in the given program is
     above a threshold $N$, it delegates to \featopt{}. Otherwise, it delegates
     to \featcon{}.}}

   \def\desch{\desc{Like \featconf{}; it delegates to \statselfopt{}
     or \statselfcon{} instead.}}

  \def\descj{\desc{Like \featconf{}; it delegates to \stattotalopt{}
     or \stattotalcon{} instead.}}

 \begin{tabular}{l l l}
    {\bf Profile} & {\bf Strategy Name} & {\bf Description}  \\ \hline
    %%%%%%%%%%%%%%%%%%%%%%%%%%%%%%%%%%%%%%%%%%%%%%%%%%%%%%%%%%%%%%%%
    \multirow[b]{2}[+19]{*}{{\em feature-specific\/}} & 
    \costoptkw{}      &   \desca           \\ \relax
    & \costconkw{}      &   \descb           \\ \relax
    & \confkw{}         &   \descg           \\ \hline

    \multirow[b]{2}[+17]{*}{{\em statistically\/} ({\em self\/})} &
    \costoptkw{}   &   \descc           \\ \relax
    & \costconkw{}   &   \desce           \\ \relax
    & \confkw{}      &   \desch           \\ \hline

    \multirow{2}[+17]{*}{{\em statistically\/} ({\em total\/})} & 
    \costoptkw{} &   \descd           \\ \relax
    & \costconkw{} &   \descf           \\ \relax
    & \confkw{}    &   \descj           % \\ \relax

 \end{tabular}

 
  \caption{The Composite Strategies of the Experiment and their Descriptions}
  \label{f:cstrategies}
\end{figure}
%% -----------------------------------------------------------------------------

\paragraph{Composite strategies.} While the basic strategies ignore the cost of
 adding types to an untyped components, developers do not. Adding types to an
 entire module in Typed Racket may impose a significant effort. Hence, the
 experiment includes composite strategies that account for this cost.

Figure~\ref{f:cstrategies} lists these composite strategies. In addition to the
 performance cost reported in the profile, the \costkw{} strategies rank the
 cost of boundaries in terms of the labor needed to equip the two components
 with types.  They give priority to those boundaries that involve components
 that are already typed.  For those, migration just means toggling their type
 enforcement regime, which is essentially no labor.  As pointed out, ripple
 effects also factor into the decisions that strategies have to make. The
 \confkw{} strategies decide whether to be \optkw{} or \conkw{} based on how
 many components are already typed.  When most components are untyped, the risk
 of a ripple effect outweighs the benefits of the \optkw{} strategy. Hence
 strategies favor a \conkw{} approach for sparsely typed programs and an
 \optkw{} strategy for densely typed ones.

\paragraph{Null Strategies} An experiment must include baselines, i.e., the
 building block for a null hypothesis. In terms of strategies, the experiment
 needs a \agnostickw{} strategy.  If this \agnostickw{ strategy is less
 successful in eliminating performance bottlenecks than the profiling ones, then
 the feedback from the profiler plays a meaningful role; otherwise, the
 experiment is meaningless.

The results presented in the next section include two rational programmers using
 \agnostickw{} strategies. The first one, \randkw{}, aims to invalidate the null
 hypothesis with random choices. Specifically, it picks a random boundary with
 types of different strictness and suggests to modify the two sides of the
 target boundary in either an \optkw{} or a \conkw{} manner.  The second
 \agnostickw{} strategy, \togglekw{}, is due to~\citet{g-deep-shallow}. It
 serves as a point of comparison with Greenman's results, which do not rely on
 profiling information. If the given program uses a mixture of shallow and deep
 enforcement, the strategy directs its rational-programmer mode to equip all
 typed components with deep enforcement. If all components already use deep, it
 instructs the mode to toggle the enforcement to shallow.




\subsection{Migration Lattices and their Navigation}
\label{subsec:lattice}
\newcommand{\gtpurl}{\url{https://docs.racket-lang.org/gtp-benchmarks/index.html}}
%% -----------------------------------------------------------------------------

Gradual type migration is an open and challenging problem for realistic
programs~\cite{rch:in-out-infer-gt, km:ts-type-evo,
mp:gt-decidable, ccew:gt-migrate, gc:gt-infer,
cagg-solver-based-migration,clps-popl-2020,js-infer,ruby-static-infer,unif-infer,
msi:gt-infer-hm,dyn-infer-ruby,profile-guided-typing,jstrace,gen-ts-decl,
lambdanet,nl2ptype,learn-types-big-data,ml-ts}. For
any untped component, a migrating programmer has to choose among several
practical type annotations and often among an infinite number of theoretical
ones. But, to make a rational-programmer experiment computationally feasible, it
is necessary to avoid this dimension.

Fortunately, the construction of the corpus of scenarios from a carefully
selected set of suitable seed programs can solve the problem. For instance, the
established GTP benchmarks~\cite{gtnffvf-jfp-2019}\footnote{\gtpurl{}} are
representative of the programming styles in the Racket world, and they come with
well-chosen type annotations for all their components.  Hence, the migration
lattices can be pre-constructed for all benchmark programs.  It is thus possible
to start a rational programmer mode from any performance-debugging scenario in
this lattice---a program with intolerable performance---and make it use the
known types of the components to chart a path through the program's migration
lattice.

A rational-programmer mode thus combines a strategy \strategyvar{} with a
function that picks points from the migration lattice. That is, the mode
$\mode{\strategyvar{}}$ is a partial function from a program $\program{}_0$ to a
program $\program{}_n$. To get from $\program{}_0$ to $\program{}_n$,
$\mode{\strategyvar{}}$ iterates \strategyvar{} on its input and applies its
recommendations to obtain the next $\program{}_i$.  In other words, $\mode{\strategyvar{}}$
constructs a \emph{migration path}, a sequence of programs ${\program{}_0},
\ldots, {\program{}_n}$ from a migration lattice. If \strategyvar{} cannot make
a recommendation at any point in this lattice, $\mode{\strategyvar{}}$ does not
produce a result. The following definitions translate these points into rigorous
language. 

\paragraph{The Migration Lattice.}  All programs $\program{}_i$ are nodes in the
\emph{migration lattice} \lattice{\program{}_u} where ${\program{}_u}$, is like
${\program{}_i}$ but all its components are untyped. The bottom element of
\lattice{\program{}_u} is ${\program{}_u}$; its top elements are
${\program{}_u}$'s fully typed versions $\program{}_t$ (with some combination of
deep and shallow). In between these extreme points are all the remaining
\emph{configurations} of $\program{}_u$. The $3^N$ configurations of
\lattice{\program{}_u} are ordered: $\ordered{\program{}_i}{\program{}_j}$ iff
$\program{}_j$ has at least one component that is untyped in
$\program{}_j$. Hence the lattice is organized in \emph{levels} of incomparable
configurations, each of which lacks types for the same components but differs
from others in the choice of enforcement regime for one (or more) of its typed
components.  $\orderqed{\program{}_i}{\program{}_j}$ denotes that either
$\program{}_i$ and $\program{}_j$ are at the same level or $\program{}_i$ is at
some level below $\program{}_j$.

A migration path corresponds to a collection of configurations $\program{}_i$,
$0 \leq i < n$, such that $\orderqed{\program{}_i}{\program{}_{i+1}}$. In fact,
either $\program{}_i$ and $\program{}_{i+1}$ are at the same level or they are
at two neighboring ones.  This statement is the formal equivalent to the
description from the preceding section that strategies either add types to a
single previously untyped component or toggle the type enforcement of existing
typed components.\footnote{None of the strategies modifies a boundary where both
side are untyped. Such boundaries are invisible to the boundary
profiler, and the strategies that use the statistical one explicitly filter them
out. Ditto for the profiler-agnostic ones.}  In other words, a migration path is
a weakly ascending chain in
\lattice{\program{}_u}.

%% MF: I think this is obvious. We can discuss if you think not. 

% This last observation is key for the implementation of the rational-programmer
% experiment. If for each $\program{}_u$ in the experiment, one of its fully type
% versions $\program{}_t$ is also available, then the construction of
% \lattice{\program{}_u} is automatic, and as a result, so are the modifications
% that the strategies produce. Therefore, given implementations of the strategies
% and a collection of seed programs with both untyped and fully typed versions,
% the rational-programmer experiment reduces to a fully automated push-button
% process that asks all modes of the rational programmer to construct a migration
% path starting at the same configurations of
% \lattice{\program{}_u}, and compares their successes. 

\paragraph{Performance-debugging scenarios and success criteria.} Completing the
formal description of the experiments demands answers to two more questions.

The first concerns the selection of the starting points for the modes of the
rational programmer, i.e., the \emph{performance-debugging scenarios}.  The
answer is that every configuration $\program$ that exhibits performance
degradation compared to $\program{}_u$ above a certain threshold is an
interesting starting point for a rational-programmer exploration.

\begin{quote} \em

Given a migration lattice \lattice{\program{}_u},
a \emph{performance-debugging scenario} is a configuration $\program$ in \lattice{\program{}_u} 
iff
\slowdown{\program}{\program{}_u} > \takikawa{},\\
 where \slowdown{\program}{\program{}_u} is the ratio of the performance of \program{} over the performance of $\program{}_u$,\\ 
 and $\takikawa{}$ is a constant that signifies the maximum acceptable performance degradation.
\end{quote}

The second question is about differentiating successful from failing migrations.
In this case, there are two answers. Strictly speaking, performance should
always improve, otherwise the programmer may not wish to invest any more effort
into migration.  In the worst case, performance might stay the same for a few
migration steps, before it becomes tolerable.

\begin{quote} \em

A migration path ${\program{}_0} \ldots {\program{}_n}$ in a lattice \lattice{\program{}_u}
is \emph{strictly successful}
iff
\begin{enumerate}
  \item $\program{}_0$ is a performance-debugging scenario,
  \item $\slowdown{\program{}_n}{\program{}_u} \leq \takikawa{}$, and 
  \item for all $0 \leq i < n$, $\slowdown{\program{}_{i+1}}{\program{}_{i}} \leq 1$.
 \end{enumerate} 
\end{quote}
The construction of a strictly successful migration path requires a strategy
that suggests modifications that monotonically improve the performance of  a
program en route to making it performant.

The alternative to strict success is to represent a programmer who tolerates
occasional setbacks. Accepting that a migration path may come with $k$ setbacks,
a $k$-loose success relaxes the requirement for monotonicity $k$ times.

\begin{quote} \em

Given $k \leq n$, a migration path ${\program{}_0} \ldots {\program{}_n}$ in a
lattice \lattice{\program{}_u} is \emph{$k$-loosely successful}
iff 
\begin{enumerate}
  \item  $\program{}_0$ is a performance-debugging scenario,
  \item $\slowdown{\program{}_n}{\program{}_u} \leq \takikawa{}$  
  \item there exists a subsequence of the path 
    ${\program{}^\prime_0} \ldots {\program{}^\prime_m}$ such that
     $m + k = n$ and
      for all $0 \leq j < m$,
      $\slowdown{\program{}_{j+1}}{\program{}^\prime_{j}} \leq 1$.
      \subitem (This subsequence is not necessarily a path in the lattice.)
  \end{enumerate} 
\end{quote}
The construction of a $k$-loose successful migration path allows a strategy to
temporarily stir migration towards worse performance. The constant $k$ is an
upper bound on the number of missteps.

A patient programmer may wish to relax this constraint even more.

\begin{quote} \em
A migration path is $N$-loosely successful iff for any $k, n$, $k \leq n$ it is
\emph{$k$-loosely successful}. 
\end{quote}



\subsection{The Experimental Questions}
\label{subsec:questions}
%% -----------------------------------------------------------------------------

Equipped with rigorous definitions, it is possible to formulate the research
questions precisely:
\begin{description}

\item[$Q_X$] How successful is a strategy $X$ with the elimination of
  performance overhead?

\item[$Q_{X/Y}$] Is strategy $X$ more successful than strategy $Y$ in this
  context?

\end{description}

Answering $Q_X$ boils down to determining the success and failures of
$X$ for all performance-debugging
scenarios in all available lattices. If, for a large number of scenarios,
$X$ charts migration paths that are strictly successful, the answer is
positive. Essentially, the large number of scenarios is evidence that when a
rational programmer reacts to profiler feedback following $X$, it is likely to
improve performance. Notably, the above
description uses the strict notion of success, which sets a high bar.
Hence, the rational programmer not only manages to tune performance at
a tolerable level but each suggestion of its strategy brings the rational
programmer closer to its target. Swapping the notion of strict success for
$k$-loose success relaxes this high standard, and offers answers to $Q_X$ when
allowing for some bounded flexibility in how well the intermediate suggestions
of $X$ help the rational programmer. For completeness, the next section also
reports the data collected for the notion of $N$-loose success.

While an answer to $Q_X$ constitutes an evaluation of a strategy $X$ for
interpreting profiler feedback in absolute terms, an answer to $Q_{X/Y}$ is
about the relative value of $X$ versus some other strategy
$Y$. This second question asks whether the proportion of scenarios
in which $X$ succeeds and $Y$ fails is higher that
the proportion of scenarios where $Y$ succeeds and
$X$ fails. Of course, the answer may not be clear cut as $X$ and $Y$
may perform equally well in most scenarios, or may have complementary success
records. But, relaxing the notion of success by different factors $k$ may
help distinguish $X$ and $Y$ based on the quality of the feedback they produce.

Importantly, when $Y$ is the $\randkw{}$ strategy and the answer to
$Q_X/\randkw$ is positive, then the experiment invalidates its null-hypothesis.
Put differently, the success of $X$ is not due to sheer luck but the rational
use of profiler feedback.

Summing up, the rational programmer process for answering $Q_X$ and $Q_{X/Y}$ rests on the following
experimental plan:
\begin{enumerate}

\item Create a large and diverse corpus of performance-debugging scenarios.

\item Calculate the migration paths for each strategy for
  each scenario.

\item Compare the successes and failures of the strategies.

\end{enumerate}

Although prior work shows that many configurations of the GTP benchmarks
run slowly, it does not answer $Q_X$ and $Q_{X/Y}$ 
---even in the $N$-loose case.  \citet{gtnffvf-jfp-2019} attempt to
investigate an $N$-loose version of  $Q_X$, but severely limit the length of paths.
\citet{g-deep-shallow} consider only paths that start from the untyped
configuration, end at a fully-typed and consist of configurations that 
all exhibit a slowdown below a threshold.  Both previous studies thus exclude
configurations that have high slowdown but can be systematically
transformed to ones with acceptable performance (say: $80x \rightarrow 70x
\rightarrow 20x \rightarrow 1x$). That said, before the rational programmer 
method it was by no means clear how to examine questions about such paths in a
principled manner.

