%% -----------------------------------------------------------------------------

Turning the sketch from section~\ref{sec:ideas} into a large-scale automated
experiment requires the formal descriptions of rational programmers. In turn,
this task demands (1) strategy components that identify (2) the next migration
step, plus (3) a collection of performance-debugging scenarios. As the preceding
section implies, parts~2 and~3 are tied together. Every rational programmer
recognizes whether a given program is a performance-debugging scenario; if so,
it applies a strategy to the profiler feedback and migrates the program---which
should yield a performant program or another performance-debugging scenario.
The preceding section also implies that a strategy's response is one of three
possibilities: (1) to signal failure; (2) to add types and to specify their
enforcement regime (deep, shallow); or (3) to toggle from one regime to
another. Hence it is possible to specify strategies independently of the
scenarios per se. Equipped with formal descriptions, it is possible to turn the
generic research question of the introduction into questions with a quantitative
nature.

This section provides precise the required formal descriptions.
Section~\ref{subsec:strategies} presents the profiling strategies.
Section~\ref{subsec:lattice} characterizes the performance-debugging scenarios
that serve as starting navigation points and how a type-based migration is a
path through a lattice of program configurations. Using these definitions, It
also lays out the criteria for successes and failures. Finally,
section~\ref{subsec:questions} formalizes the precise experimental questions and
the experimental procedure that answers them.

%% -----------------------------------------------------------------------------

\def\exp#1#2{\subsection{#2} \label{subsec:#1} \input{experiment-#1.tex}}

\exp{strategies}{The Profiling Strategies}
\exp{lattice}{Migration Lattices and their Navigation}
\exp{questions}{The Experimental Questions}
