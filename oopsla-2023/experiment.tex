
\subsection{tmp: definitions}

A benchmark is a family of programs.
There are $3^N$ program configurations (\emph{configurations} for short)
in each family, where $N$ is the number of \emph{migratable} modules.
These modules can depend on any number of additional \emph{contextual} modules.

A migration target (\emph{target}) is a set of modules that should be equipped
with types.
There are $2^N$ targets per benchmark.


\subsection{Data Collection}

% https://www.cloudlab.us/

Our experiments call for three kinds of data for every configuration across the benchmarks:
\begin{enumerate}
  \item Running time
  \item Boundary-profile output
  \item Statistical-profile output
\end{enumerate}

To collect one running time, we used nine runs of the configuration:
one throwaway run to warm up the Racket JIT and eight other runs to compute an average.
To collect boundary and statistical output, we ran each configuration once.
Thus we took $11 * \totalnumconfigs{} = \totalnummeasurements{}$ measurements in total.

We collected all the data on CloudLab~[CITE].
For running times, we used the Wisconsin cluster c220g1 
FILL: exception for morsecode.

%% 2023-03-22 bg: quadT ran on m510 nodes

boundary from another cluster, same racket
exceptions: dungeon, morsecode
profile, ditto

\begin{table}[t]
  \caption{Datasets}
  \label{t:data-collection}

  \begin{tabular}{ll}
    Dataset & Where from \\\midrule
    dungeon runtime & \machinename{c220g2} Racket FILL TR \commitname{29ea3c10}{29ea3c105e0bd60b88c1fd195b54fa716863f690} \\
    dungeon b + s?? & TBD \\
    morsecode ? & \machinename{TBD} Racket FILL TR cherry-pick TBD \\
    %% quadT runtime & \machinename{m510}, standard Racket \\
    other runtime & \machinename{c220g1} Racket FILL TR FILL \\
    other boundary & \machinename{m510}, Racket FILL \\
    other statistical & \machinename{m510}, Racket FILL
  \end{tabular}

  \bigskip

  \begin{tabular}{ll}
    Machine Type & Location, RAM, CPU speed, ... \\\midrule
    \machinename{c220g1} & FILL \\
    \machinename{c220g2} & FILL \\
    \machinename{m510} & FILL
  \end{tabular}
\end{table}

