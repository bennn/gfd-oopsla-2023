%% -----------------------------------------------------------------------------

Turning the sketch from section~\ref{sec:ideas} into a large-scale automated
experiment requires formal descriptions: the modes of the rational programmer;
the strategies that create modes from the rational programmer; and the notion of
debugging scenario. A strategy component identifies the next migration step.
Every mode of the rational programmer recognizes whether a given program is a
performance-debugging scenario; if so, it applies a strategy to the profiler
feedback and migrates the program---which should yield a performant program or
another performance-debugging scenario.  The preceding section also implies that
a strategy's response is one of three possibilities: (1) to add types and to
specify their enforcement regime (deep, shallow); (2) to toggle from one regime
to another; or (3) to fail to act. Hence it is possible to specify strategies
independently of the scenarios per se. Equipped with formal descriptions, it is
possible to turn the generic research question of the introduction into
questions with a quantitative nature.

This section provides the required formal descriptions.
Section~\ref{subsec:strategies} presents the profiling strategies.
Section~\ref{subsec:lattice} characterizes the starting navigation points
(performance-debugging scenarios)
and how a type-based migration is a
path through a lattice of program configurations. Using these definitions, It
also lays out the criteria for successes and failures. Finally,
section~\ref{subsec:questions} formalizes the precise experimental questions and
the experimental procedure that answers them.

%% -----------------------------------------------------------------------------

\def\exp#1#2{\subsection{#2} \label{subsec:#1} \input{experiment-#1.tex}}

\exp{strategies}{The Profiling Strategies}
\exp{lattice}{Migration Lattices and their Navigation}
\exp{questions}{The Experimental Questions}
