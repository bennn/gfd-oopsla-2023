%% -----------------------------------------------------------------------------

Turning the sketch from section~\ref{sec:ideas} into a large-scale
automated experiment requires formal descriptions for both the profiling
strategies of the rational programmer and the notion of debugging
scenario. As the preceding section discusses, given a scenario, a strategy
identifies the next migration step, which should yield either an acceptable
program or another performance-debugging scenario.  The preceding section
also implies that the migration step is one of three possibilities: (1)
to add types and to specify their enforcement regime (deep, shallow); (2)
to toggle from one regime to another; or (3) to fail to act. Hence it is
possible to specify strategies independently of the scenarios per se.
Equipped with formal descriptions, it is possible to turn the generic
research question of the introduction into questions with a quantitative
nature.

\Cref{subsec:strategies} presents the profiling strategies.
\Cref{subsec:lattice} characterizes performance-debugging
scenarios, which act as starting navigation points,
and how a type-based migration is a path
through a lattice of program configurations.
It also lays out the criteria for successes and failures for strategies.
Finally, \cref{subsec:questions} formalizes the precise experimental
questions and the experimental procedure that answers them.

%% -----------------------------------------------------------------------------

\def\exp#1#2{\subsection{#2} \label{subsec:#1} \input{experiment-#1.tex}}

\exp{strategies}{The Profiling Strategies}
\exp{lattice}{Migration Lattices and their Navigation}
\exp{questions}{The Experimental Questions}
