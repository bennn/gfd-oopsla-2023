For our experiments, we consider every configuration as a starting point.
From the starting point, the goal is to reach a fast configuration without
removing types from a module.
(In other words, the \emph{target} is the set of typed modules in the start
configuration.)
A fast configuration runs at least as quickly as the untyped code.
Using the established terminology~\cite{vss-popl-2017,bbst-oopsla-2017},
we instatiate the Takikawa constant to 1x ($T=1$).

A very simple way to try improving performance is to toggle between
Deep and Shallow.
\citet{g-pldi-2022} provides justification.
Any configuration can reach 0, 1, or 2 other configurations by
toggling, that is, by changing all its typed modules to Deep or
changing all to Shallow.
(Only the untyped configuration can reach 0 others.)

First question: for how many configurations does $T=1$ present a debugging challenge?
More precisely, how many cannot reach $T=1$ by toggling?

(The untyped configuration can trivially reach $T=1$.
The fully-typed configuration can usually reach $T=1$, unless there are heavy
boundaries to untyped contextual modules.)

\Cref{t:baseline-trouble} shows that many benchmarks need help,
over 100k configurations in total.
The median \% that need help is 82\%.
Only \bmname{fsm} and \bmname{lnm} have low percentages: 8\% and 44\%.
FILL say more

\begin{table}[t]
  \caption{How many of the $3^N$ configurations need help to reach $T=1$?}
  \label{t:baseline-trouble}
  \begin{tabular}{lrr}
    Benchmark  & Count & Pct \\\midrule
    sieve      & 8     & 88.89\% \\
    morsecode  & \ycell{55}    & \ycell{67.90\%} \\
    forth      & 74    & 91.36\% \\
    fsm        & \ycell{36}    & \ycell{44.44\%} \\
    fsmoo      & \ycell{50}    & \ycell{61.73\%} \\
    mbta       & 72    & 88.89\% \\
    zombie     & 64    & 79.01\% \\
    dungeon    & 242   & 99.59\% \\
    jpeg       & \ycell{168}   & \ycell{69.14\%} \\
    lnm        & \ycell{57}    &  \ycell{7.82\%} \\
    suffixtree & 584   & 80.11\% \\
    kcfa       & 1799  & 82.26\% \\
    snake      & 6560  & 99.98\% \\
    take5      & 6558  & 99.95\% \\
    acquire    & 19495 & 99.04\% \\
    tetris     & 14879 & 75.59\% \\
    synth      & 58022 & 98.26\% \\
  \end{tabular}

\end{table}

\begin{figure}[t]
  \caption{Distribution of trouble configurations by size of the migration target.}
  \label{f:where-trouble}
  \includegraphics{data/where-trouble.pdf}
\end{figure}

FILL \cref{f:where-trouble}
... delete this figure?


\begin{figure}[t]
  \includegraphics[width=\textwidth]{data/cdf-overhead_boundary.pdf}
  \caption{f:overhead-bnd}
  \label{f:overhead-bnd}
\end{figure}

\begin{figure}[t]
  \includegraphics[width=\textwidth]{data/cdf-overhead_prf_self.pdf}
  \caption{f:overhead-self}
  \label{f:overhead-self}
\end{figure}

\begin{figure}[t]
  \includegraphics[width=\textwidth]{data/cdf-overhead_prf_total.pdf}
  \caption{f:overhead-total}
  \label{f:overhead-total}
\end{figure}

\begin{figure}[t]
  \includegraphics[width=\textwidth]{data/cdf-steps_boundary.pdf}
  \caption{f:steps-bnd}
  \label{f:steps-bnd}
\end{figure}

\begin{figure}[t]
  \includegraphics[width=\textwidth]{data/cdf-steps_prf_self.pdf}
  \caption{f:steps-self}
  \label{f:steps-self}
\end{figure}

\begin{figure}[t]
  \includegraphics[width=\textwidth]{data/cdf-steps_prf_total.pdf}
  \caption{f:steps-total}
  \label{f:steps-total}
\end{figure}

\clearpage

