\newcommand{\numgtp}{16}
\newcommand{\gtpurl}{\url{https://docs.racket-lang.org/gtp-benchmarks/index.html}}
\newcommand{\boundaryMB}{536} % 536244 kb
\newcommand{\statisticalMB}{4645} % 4645788 kb
\newcommand{\runtimeMB}{44}

This section presents the results of a rational programmer
experiment that compares 17 strategies on programs from
the GTP benchmark suite~\cite{gtp-benchmarks}.


\subsection{Experiment}

\subsubsection{Benchmarks}

The GTP benchmarks are a collection of small to mid-sized Typed Racket programs
that support a full lattice of mixed-typed interactions.
Since the benchmarks originate from useful scripts
and exercise a variety of types at module boundaries~\cite{gtnffvf-jfp-2019},\footnote{\gtpurl{}}
they are an appropriate testbed for comparing navigation strategies.
Configurations with high overhead due to deep types should benefit from
feature-specific profiling, while configurations that suffer from shallow
types may benefit more from the statistical profiler.

The experiment uses the v7.0 release of the GTP Benchmarks with small modifications
to support feature-specific profiling.
Version 7.0 is outdated, but was the latest version when we began collecting
data in Summer 2022~(\cref{sec:data}).
The modifications add static information so that the profiler can peek through
\emph{adaptor modules} in specific benchmarks.
They do not change the run-time behavior of the code.
In total, nine benchmarks are affected~(\cref{sec:adaptor-rewrite}).
Source code for the original benchmarks and our modified ones will be available
in the artifact for this paper.

We omit four of the 21 benchmarks:
\bmname{zordoz} because it currently cannot run all deep/shallow/untyped configurations
due to a known issue;\footnote{\url{https://github.com/bennn/gtp-benchmarks/issues/46}}
\bmname{gregor},\bmname{quadT}, and \bmname{quadU} because each has over 1.5
million configurations and we lack the resources to run a full experiment;\footnote{
\textbf{To OOPSLA reviewers}: we have collected performance data for
\bmname{quadT} (which has the worst performance of the three) and hope to
collect profile results in the coming weeks.  We have no plans to measure
\bmname{gregor} or \bmname{quadU}.
}
and \bmname{sieve} because it is very small, with only four configurations.


\subsubsection{Profiling Tools}

Racket ships with tools for statistical~[CITE] and feature-specific
profiling~\cite{}.
The feature-specific profiler is aware of contracts as a default feature.
Since deep gradual types compile to contracts,
Since deep gradual types compile to contracts, this default contract
mode is suitable for the experiment.

The experiment uses slightly-modified variants of these default tools
to make their output easier to analyze.
First, the modified statistical profile renders its
findings as a JSON graph rather than as a table~(see~\cref{f:fsm-code:statistical}).
Second, the modified feature-specific profiler groups contracts
by boundary (using the two sides of their blame objects) and sorts groups
in descending order.
Thus, the most expensive \emph{boundary} appears at the top rather than the single
most expensive contract~(shown in~\cref{f:fsm-code:boundary}).


\subsubsection{Data}
\label{sec:data}

The ground-truth data for our experiment consists of running times,
feature-specific profile output, and statistical profile output.
Collecting this data required three basic steps for each configuration
of the \numgtp{} benchmarks:
\begin{enumerate}
  \item
    Run the configuration once, ignoring the result, to warm up the JIT.
    Then run eight more times measuring performance.
    In particular, we recorded {cpu time}s as reported by the Racket
    \code{time} function.
  \item
    Install the modified feature-specific profiler and run it once,
    collecting output in a file.
  \item
    Install the modified statistical profiler and run it once, collecting output.
\end{enumerate}
Complications arose because of the large scale of the experiment, and because
we discovered and fixed two issues in Typed Racket along the way.


\begin{table}[ht]
  \caption{Datasets, their origin, and server details}
  \label{t:data-collection}

  \begin{tabular}{llll}
    Dataset           & Server & Racket & Typed Racket \\\midrule
    dungeon           & \machinename{c220g2} & \stdrkt{} &  \commitname{29ea3c10}{29ea3c105e0bd60b88c1fd195b54fa716863f690} \\
    morsecode         & \machinename{m510}   & same & \commitname{700506ca}{700506ca01393f211229101c47d8420f6d535de9} (cherry pick) \\
 %% quadT runtime  & \machinename{m510}   & same & same \\
    other runtime     & \machinename{c220g1} & same & default \\
    other profile      & \machinename{m510}   & same & same
  \end{tabular}

  \bigskip

  \begin{tabular}{llrrr}
    %% multi-cpu, multi-core machines .. but we didn't use that, right?
    Server & Site & CPU Speed & RAM & Disk \\\midrule
    \machinename{c220g1} & Wisconsin & 2.4GHz & 128GB & 480GB SSD \\
    \machinename{c220g2} & Wisconsin & 2.6GHz & 160GB & 480GB SSD \\
    \machinename{m510}   & Utah      & 2.0GHz &  64GB & 256GB SSD
  \end{tabular}
\end{table}

In total, we analyzed 116,154 configurations
and recorded 1,277,694 measurements.
%% $11 * \totalnumconfigs{} = \totalnummeasurements{}$
All collection took place on CloudLab~\cite{cloudlab} servers using a recent
%% https://github.com/racket/typed-racket/pull/1316
version of Racket (as of July 2022).
We utilized one core per server in an effort to obtain reliable performance
measurements.
The dataset takes up 5GB overall, most of which is due to the statistical
profiler (4.6GB).
Measurements began in July 2022 and have continued to the present (April 2023)
because we are working on the 4.7 million configurations of \bmname{quadT}.

Initially, we planned to collect running times for all configurations including
those of \bmname{quadT} before moving on to the profilers.
Due to this ambitious plan and the popularity of resources on CloudLab,
we spread our experiment across three clusters.
\Cref{t:data-collection} (top) matches datasets to clusters.
Most running times came from \machinename{c220g1} machines managed by the University of Wisconsin-Madison.
Most profiler results (feature-specific and statistical) came from \machinename{m510} machines
managed by the University of Utah.
The specifications of these machines are reported at the bottom of~\cref{t:data-collection}.

Both the \bmname{dungeon} and \bmname{morsecode} benchmarks took a different
path and used only one machine type: \machinename{c220g2} and
\machinename{m510}, respectively.
This is because we collected their data much later than the other benchmarks,
after discovering and fixing two issues in Typed Racket that significantly
affected their performance.\footnote{\url{https://github.com/racket/typed-racket/pull/1282}, \url{https://github.com/racket/typed-racket/pull/1316}}
For \bmname{dungeon}, we used a newer Typed Racket commit.
For \bmname{morsecode}, we cherry-picked one small commit onto the default Typed Racket
for Racket v8.6.0.2 (to be available in our artifact).
Fixing these issues \emph{was not necessary} for the rational programmer experiment.
The only reason for re-running was to stay current with major changes to Typed Racket.



\begin{table}[ht]
  \caption{How many of the $3^N$ configurations have any overhead to begin with?}
  \label{t:baseline-trouble}
  \begin{tabular}[t]{l@{\qquad}l}
    \begin{tabular}[t]{lrr}
      Benchmark           & $3^N$ & \% Interesting \\\midrule
      \bmname{morsecode}  &    81 & 82.72\% \\
      \bmname{forth}      &    81 & 93.83\% \\
      \ycell{\bmname{fsm}}        &    \ycell{81} & \ycell{76.54\%} \\
      \bmname{fsmoo}      &    81 & 83.95\% \\
      \bmname{mbta}       &    81 & 88.89\% \\
      \bmname{zombie}     &    81 & 91.36\% \\
      \bmname{dungeon}    &   243 & 99.59\% \\
      \bmname{jpeg}       &   243 & 94.65\% \\
    \end{tabular}
    &
    \begin{tabular}[t]{lrr}
      Benchmark           & $3^N$ & \% Interesting \\\midrule
      \ycell{\bmname{lnm}}        &   \ycell{729} & \ycell{40.47\%} \\
      \bmname{suffixtree} &   729 & 98.49\% \\
      \bmname{kcfa}       &  2,187 & 92.87\% \\
      \bmname{snake}      &  6,561 & 99.97\% \\
      \bmname{take5}      &  6,561 & 99.95\% \\
      \bmname{acquire}    & 19,683 & 99.23\% \\
      \bmname{tetris}     & 19,683 & 95.47\% \\
      \bmname{synth}      & 59,049 & 99.99\%
    \end{tabular}
  \end{tabular}
\end{table}

\subsection{Interesting Scenarios}

For our experiment, we consider every configuration as a potential
performance debugging scenario.
The goal for each scenario is to find another configuration
with at least as many typed modules that runs no slower than the
untyped configuration.

Naturally, some configurations are \emph{uninteresting} as starting
points because they already run fast enough.
The untyped configuration is one of these by definition.
Uninteresting configurations are important to have because they
are the possible endpoints for a successful navigation.
But if all or most configurations are uninteresting, then
navigation becomes trivial.

There should ideally be many interesting points in a benchmark
and a handful of uninteresting ones to navigate to.
\Cref{t:baseline-trouble} tabulates this landscape for the benchmarks.
All but two have plenty of interesting scenarios: over 80\% of
all configurations have some overhead relative to untyped.
Navigating to a successful configuration is therefore a challenge
for the rational programmer.

The other two benchmarks, \bmname{fsm} and \bmname{lnm}, have
fewer interesting scenarios.
In \bmname{fsm} the number is somewhat low, at 76\% (62 configurations),
which raises some doubt about its usefulness as a benchmark.
In \bmname{lnm} the situation is worse.
Only 40\% of the configurations are interesting; the rest run within a 1x
overhead.
Navigation is not a serious issue for \bmname{lnm}; its overall
performance is good.



\subsection{Running a Rational Programmer}

We have instantiated the 17 strategies from~\cref{subsec:strategies} as
programs that navigate toward fast configurations.
Fifteen strategies use profile information as described in~\cref{f:bstrategies,f:cstrategies}.
The remaining two are profile agnostic: random boundary and toggling.

These programs navigate relentlessly toward a successful configuration,
accepting any number of setbacks~($N$-loose,~\cref{subsec:questions}).
At each step, they consult the performance data for the current configuration.
If it is not fast enough, they follow a profiler in accordance with a strategy.
For example, the \emph{conservative, statistically (self)} rational programmer
uses the ``self\%'' running times reported by the statistical profiler to
identify a costly boundary, and then converts the boundary to use shallow types.
When analyzing this data, we impose limits such as strict success and
$2$-loose success by dropping a suffix of the recorded trails.

\Cref{f:strategy-overall} presents the results of running all strategies on all
interesting scenarios.
This data lets us answer the first (parameterized) research question:

\begin{itemize}
  \item[$Q_X$] How successful is a strategy $X$ with the elimination of
    performance problems?
\end{itemize}

Each stacked bar in the ``skyline'' of \cref{f:strategy-overall} provides
several answers to the question for increasingly-loose notions of success.
The lowest, widest part of each bar counts how many configurations
achieved a strict success.
The next three stories count $1$-loose, $2$-loose, and $3$-loose successes.
The striped spire counts $N$-loose successes.
The antenna uses an even weaker notion, strict 3x success, which we discuss below~(\cref{s:antenna}).

The strategies are successful in 2\% of configurations in the worst case and
55\% in the best case:
\begin{itemize}
  \item
    \emph{Optimistic} navigation performs well when guided by the \emph{feature-specific} profiler,
    finding strict success in almost 40\% of all configurations.
    With a $2$-loose relaxation, success rises to above 50\%.
    The results are far worse, however, with \emph{statistical (total)} or \emph{statistical (self)}
    profiles, both of which perform comparably.

  \item
    \emph{Cost-aware optimistic} is almost as successful as optimistic when driven
    by \emph{feature-specific} profiles and equally successful with \emph{statistical (total)}
    and \emph{statistical (self)}.

  \item
    \emph{Conservative} navigation is unsuccessful no matter what profiler it uses.

  \item
    \emph{Cost-aware conservative} is unsuccessful as well.
    Even with $N$-loose relaxation, only a handful of configurations (2\%) achieve success.

  \item
    \emph{Configuration-aware optimistic} navigation with
    \emph{feature-specific} profiles succeeds in approximately 36\% of all
    configurations under strict and just ovef 50\% with $3$-loose.
    With \emph{statistical (total)} and \emph{statistical (self)} profiles,
    the success rate drops to 10\% even for $N$-loose.

  \item
    \emph{Random boundary} succeeds in 5\% of all configuration.
    This is a low number, but much better than the conservative strategies.
    Allowing for 1,2,3-loose success improves the rate incrementally, and an $N$-loose search
    succeeds in nearly 40\% of configurations.
  \subitem
    The results for random boundary are the average success rates across three trials. The standard deviations
    for each number were low, under $0.10$.

  \item
    \emph{Toggling} achieves strict success a bit more often than random, in roughly 6\% of all configurations.
    The other notions of success do not apply to toggling because it can step to at most one configuration.
\end{itemize}

These findings provide a coarse answer to our second research question~(\cref{s:hh} gives a detailed comparison):

\begin{itemize}
\item[$Q_{X/Y}$] Is strategy $X$ more successful than strategy $Y$ in this
  context?
\end{itemize}

\begin{figure}[ht]
  \includegraphics[width=\columnwidth]{data/strategy-overall-feasible.pdf}
  \caption{How many of the 114,428 scenarios does each strategy succeed in, for six notions of success.}
  \label{f:strategy-overall}
\end{figure}

\emph{Optimistic, feature-specific} navigation is the most likely to
succeed on an arbitrary configuration.
\emph{Cost-aware} and \emph{configuration-aware} using the optimistic strategy
are close behind.
The conservative strategies are least likely to find a successful configuration
no matter what profiler they use.
Feature-specific profiling is always more successful than statistical profiling.

\subsubsection{Antenna: 3x Strict Success}
\label{s:antenna}

There are two possible reasons for the poor performance of the conservative
strategies.
One is that they are entirely unproductive: they lead to worse performance, end
of story.
The other possibility is that they do improve performance, but are unable
to achieve a 1x overhead because there are no such configurations with
mostly shallow types.
This second possibility is likely due to the current implementation of shallow
types~\cite{g-deep-shallow}, which almost always add some overhead (but much
less than deep types).

The bars in \cref{f:strategy-overall}, however, do not show whether conservative
is improving or degrading performance.
Hence the thin lines above each bar (antenna), which count configurations that see
a strict success when a 3x slowdown is acceptable.
The number 3x is the classic Takikawa constant for ``acceptable'' gradual typing
overhead~\cite{tfgnvf-popl-2016,vss-popl-2017,bbst-oopsla-2017}.
We use it to give a small amount of wiggle room.
Changing to 2x or 4x does not significantly change the outcome.

For \emph{conservative} and \emph{cost-aware conservative}, allowing a 3x overhead
improves results across the profilers.
An additional 10\% of configurations succeed.
%% TODO how many are immediate with 3x?
While some of these configurations are uninteresting scenarios at the 3x level,
others are not; thus, conservative can improve performance.

The optimistic strategies with statistical profiles improve in a similar way
for 3x success.
Optimistic with feature profiles does not improve.
This means that every configuration that can achieve a 3x success with statistical
profiles is able to achieve a 1x success with feature-specific profiles.

Random boundary does not improve for 3x strict success, likely because it chooses
an unlucky boundary and fails the ``strict'' criterion.
Toggling improves tremendously for 3x success.
This is in line with prior work on shallow, which reports a median worst-case overhead
of 4.2x on the GTP Benchmarks~\cite{g-deep-shallow}.
Evidently, about 45\% of configurations can reach a 3x overhead by switching to
shallow types.


\begin{table}[ht]
  \caption{How many scenarios can possibly reach 1x without removing types?}
  \label{t:blackhole}
  \begin{tabular}[t]{l@{\qquad}l}
    \begin{tabular}[t]{lrr}
      Benchmark                &  \# Scenario &  \% Hopeful \\\midrule
      \bmname{morsecode}       &           67 &    100\% \\
      \bmname{forth}           &           76 &     36.84\% \\
      \bmname{fsm}             &           62 &    100\% \\
      \bmname{fsmoo}           &           68 &    100\% \\
      \rcell{\bmname{mbta}}    &   \rcell{72} & \rcell{0\%} \\
      \bmname{zombie}          &           74 &     35.14\% \\
      \rcell{\bmname{dungeon}} &  \rcell{242} & \rcell{0\%} \\
      \bmname{jpeg}            &          230 &    100\%
    \end{tabular}
    &
    \begin{tabular}[t]{lrr}
      Benchmark                &   \# Scenario &  \% Hopeful \\\midrule
      \bmname{lnm}             &           295 &    100\% \\
      \bmname{suffixtree}      &           718 &    100\% \\
      \bmname{kcfa}            &         2,031 &    100\% \\
      \bmname{snake}           &         6,559 &    100\% \\
      \rcell{\bmname{take5}}   & \rcell{6,558} & \rcell{0\%} \\
      \bmname{acquire}         &        19,532 &      5.45\% \\
      \bmname{tetris}          &        18,791 &    100\% \\
      \bmname{synth}           &        59,046 &    100\%
    \end{tabular}
  \end{tabular}
\end{table}

\begin{figure}[ht]
  \includegraphics[width=\columnwidth]{data/strategy-overall-hopeful.pdf}
  \caption{How many of the 88,992 hopeful scenarios does each strategy succeed in, for six notions of success.}
  \label{f:strategy-hope}
\end{figure}

\subsubsection{Omitting Hopeless Scenarios}

Some of the \emph{interesting} scenarios in the experiment cannot possibly
achieve success given the rules that the rational programmer must follow.
In particular, the rational programmer cannot remove types.
If a configuration cannot reach a point with 1x overhead except by removing
types, then its scenario is a hopeless one.
It might be the case that hopeless scenarios explain the low success rates
in~\cref{f:strategy-overall}.

\Cref{t:blackhole} lists the number of interesting scenarios in each benchmark
and the percent of these that can possibly reach a successful configuration.
A low percent in the third column (\% Hopeful) of this table means that
the experiment is stacked against the rational programmer.
For several benchmarks, this is indeed the case.
Worst of all are \bmname{mbta}, \bmname{dungeon}, and \bmname{take5},
which have zero hopeful scenarios.
Three others are only marginally better: \bmname{forth}, \bmname{zombie}, and
\bmname{acquire}.

In each of these hopeless benchmarks, the fully-deep configuration runs
slower than the untyped configuration.
The difference is small (<1 second), but consistent;
the average running times for these configurations are outside one standard
deviation of one another.
Becaus of this failure at the top of the lattice, it follows that other
configurations are unsuccessful too.
Less-typed configurations incur more overhead due to mixed-typed interactions.

\Cref{f:strategy-hope} therefore takes a second look at the rational programmer
results, focusing only on hopeful scenarios.
If a configuration starts with some overhead but cannot reach a 1x point, it
is does not affect the counts.
As before, the results for \emph{random boundary} are the average across three
runs.
The standard deviation among numbers was low, but slightly higher than before ($<0.12$).

For the optimistic strategies, the results are much better.
An additional 10\% of configurations achieve strict success
with the \emph{optimistic, feature-specific} strategy, and
the $N$-loose results improve along the same lines.
With statistical profiles, the optimistic strategies improve
slightly.

Unfortunately for conservative, it performs no better.
Very few hopeful scenarios achieve an $N$-loose success.
Strict 3x successes are rare as well.
Indeed the antennae in~\cref{f:strategy-overall} are shorter than the
antennae in~\cref{f:strategy-hope}.
This means that conservative succeeded in the 3x sense on a number
of hopeless configurations. Such configurations cannot reach a 1x
point, but can reach a 3x point.


\begin{figure}[ht]
  \includegraphics[width=0.9\columnwidth]{data/head-to-head.pdf}
  \caption{Optimistic vs. the rest, strict success: losses (red bars) and wins (green bars) on all interesting scenarios.}
  \label{f:head-to-head}
\end{figure}

\subsection{Head to Head: Comparing Strategies}
\label{s:hh}

The results so far show that the \emph{optimistic} strategy
with \emph{feature-specific} profiles (opt-fsp) achieves the highest rate
of success.
An unanswered question is whether there are particular cases in which
the other strategies succeed and opt-fsp fails.

Based of \cref{f:strategy-overall}, it seems likely that opt-fsp will do
worse in some scenarios.
It fails to achieve strict success for over half the starting points;
perhaps statistical profiling or another navigation strategy succeeds
for these points.

\Cref{f:head-to-head} thus compares opt-fsp to the other strategies.
The $y$-axis counts configurations.
The $x$-axis lists all strategies including opt-fsp (on the left).
For each strategy, there are at most two vertical bars.
A red bar appears when the other strategy succeeds on configurations
where opt-fsp fails.
A green bar appears for the reverse situation, where opt-fsp succeeds
but the other fails.
Ties do not count.
The red and green bars do not combine to 100\% because there are always configurations
in which both strategies succeed or both fail.

At a glance, the tiny red bars and tall green bars give a negative answer to our question.
Other strategies rarely succeed where opt-fsp fails.
This is true even for the composite strategies \emph{cost-aware} and \emph{configuration-aware}.
For all but a few configurations, opt-fsp is the winner.
