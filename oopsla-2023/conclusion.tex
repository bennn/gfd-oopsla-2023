
%% -----------------------------------------------------------------------------

Sound migratory typing comes with several advantages~\cite{lgfd-icfp-2021,lgfd-icfp-2023} 
but also poses a serious performance-debugging challenge to
programmers who wish to use it.
Profiling tools are designed to overcome performance problems, but the
use of such tools requires an effective strategy for interpreting their output.
This paper reports
on the results of using the novel rational programmer  method to
systematically test the pragmatics of five competing strategies that use two
off-the-shelf profilers.

At the object level, the results deliver several insights:
\begin{enumerate}
  \item
    The boundary profiler works well if used with any ``optimistic''
    interpretation strategy. That is, programmers should eliminate the hottest
    boundary, as identified by the boundary profiler, by making both modules
    use deep types.

  \item
    If a program comes with a low overhead for all mixed-typed
    variants, the statistical profiler works reasonably well; otherwise the
    statistical profiler is unhelpful for performance-debugging problems in
    this context.

  \item
    While profiling tools help with debugging
    performance, the data also clarifies that for
    certain kinds of programs, the migration lattice contains a huge region
    of ``hopeless'' scenarios in which no strategy can succeed.
    These regions call for fundamental improvements to deep and shallow checks.

  \item
    Finally, the results weaken \citet{g-deep-shallow}'s report that
    adding shallow type enforcement is helpful.
    While toggling to shallow can reduce costs to a 3x overhead in many
    configurations~(\cref{f:strategy-overall}), the poor results for the
    configuration-aware strategies indicate that it is not a useful stepping
    stone toward performant (1x) configurations.
    If parity with untyped code is the goal, deep types are the way to go.
\end{enumerate}

At the meta level, the experiment once again confirms the value of the
rational programmer method. Massive simulations of satisficing rational
programmers deliver indicative results that clearly contradict anecdotal reports
of human programmers. As mentioned, a rational programmer is {\em not\/} an
idealized human developer. It remains an open question whether and how the
results apply to actual performance-debugging scenarios when human beings are
involved.

Finally, the rational programmer experiment also suggests several ideas for
future research. First, the experiment should be reproduced for alternative
mixed-typed languages. Nothing else will confirm the value of the optimistic
strategy and the boundary profiler.
It may also be the case that JIT technology, as demonstrated
in Grace~\cite{rmhn-ecoop-2019} and Reticulated~\cite{vsc-dls-2019}, drastically improves the value
of the conservative strategy and statistical profiler.
Second, the
experiment clearly demonstrates that existing profiling tools are not enough to
overcome the performance challenges of sound migratory typing. Unless researchers
can construct a performant compiler for a production language with sound
types, the community must design better profiling tools to guide
type migration.



% future work:

% - shallow fsp?

% - how this all changes when the compiler knows more optimizations 

% - rational programmer: experiment should be repeated in other settings


% Shallow / transient gets no benefit from boundary profiling,
% some benefit from statistical.
% How about a shallow FSP?


%% MF what does this mean? 
%% Would a simpler approach work just as well [CITE vmil]?

\subsubsection*{Data Availability Statement}

The data for this paper is available on Zenodo,
along with scripts for reproducing the experiment
and analyzing the results~\cite{gdf-artifact-2023}.

