\section{Modifications to the GTP Benchmarks}
\label{s:adaptor-rewrite}

To support a rational programmer experiment using feature-specific profiling,
we had to change nine of the GTP Benchmarks in a minor way.
The change lets the profiler peek through \emph{adaptor modules}.
Adaptor modules are a layer of indirection that lets benchmark with
generative types (structs)support a lattice of mixed-typed
configurations~\cite{gtp-benchmarks,tfgnvf-popl-2016}.
The benchmarks are:
\bmname{acquire}, \bmname{kcfa},
\bmname{snake}, \bmname{suffixtree}, \bmname{synth}, \bmname{take5},
\bmname{tetris}, and \bmname{zombie}.
%% FILL also quadT

In short, the trouble with an adaptor is that its name appears
in contracts instead of the name of its clients.
If one adaptor has three clients, then profiling will attribute costs to one
library--adaptor boundary instead of three library--client boundaries.

The change is to add client-specific submodules to each adaptor.
Using an adaptor with three clients as an example, the changes are:
\begin{enumerate}
  \item
    define generative types at the top level of the adaptor;
  \item
    export the generative types \emph{unsafely}, without any contract;
  \item
    create three submodules, one for each client, each of which imports
    the generative types and provides them safely; and
  \item
    modify the clients to import from the newly-created submodules rather
    than the top level.
\end{enumerate}
%
The submodules attach client-specific names to contracts without changing
run-time behavior.


\section{Benchmark Skylines}
\label{s:bm-sky}

\begin{figure}[t]
  \newcommand{\kkrow}[1]{\bmname{#1} \\ \includegraphics[width=0.39\columnwidth]{data/sky/#1-feasible.pdf}}
  \begin{tabular}[t]{ll}
    \begin{tabular}[t]{l}
      \kkrow{morsecode} \\
      \kkrow{forth} \\
      \kkrow{fsm} \\
      \kkrow{fsmoo} \\
      \kkrow{mbta} \\
      \kkrow{zombie} \\
      \kkrow{dungeon} \\
      \kkrow{jpeg} \\
    \end{tabular}
    &
    \begin{tabular}[t]{l}
      \kkrow{lnm} \\
      \kkrow{suffixtree} \\
      \kkrow{kcfa} \\
      \kkrow{snake} \\
      \kkrow{take5} \\
      \kkrow{acquire} \\
      \kkrow{tetris} \\
      \kkrow{synth}
    \end{tabular}
  \end{tabular}

  \caption{FILL skyline}
  \label{f:skyline:bm}
\end{figure}


\section{Benchmark Comparisons: Opt-FSP}

TBA

\begin{figure}[t]
  \newcommand{\hhrow}[1]{\bmname{#1} \\ \includegraphics[width=0.39\columnwidth]{data/h2h/#1.pdf}}
  \begin{tabular}[t]{ll}
    \begin{tabular}[t]{l}
      \hhrow{morsecode} \\
      \hhrow{forth} \\
      \hhrow{fsm} \\
      \hhrow{fsmoo} \\
      \hhrow{mbta} \\
      \hhrow{zombie} \\
      \hhrow{dungeon} \\
      \hhrow{jpeg} \\
    \end{tabular}
    &
    \begin{tabular}[t]{l}
      \hhrow{lnm} \\
      \hhrow{suffixtree} \\
      \hhrow{kcfa} \\
      \hhrow{snake} \\
      \hhrow{take5} \\
      \hhrow{acquire} \\
      \hhrow{tetris} \\
      \hhrow{synth}
    \end{tabular}
  \end{tabular}

  \caption{FILL Head to head, per benchmark}
  \label{f:h2h:bm}
\end{figure}
