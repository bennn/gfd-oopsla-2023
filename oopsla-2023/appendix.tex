\section{Modifications to the GTP Benchmarks} \label{s:adaptor-rewrite}

To support a rational programmer experiment using boundary profiling,
nine of the GTP Benchmarks required a minor reorganization.
The change lets the profiler peek through \emph{adaptor modules},
which are a technical device used in the benchmarks.
Adaptor modules are a layer of indirection that lets benchmarks with
generative types (i.e., Racket structs) support a lattice of mixed-typed
configurations~\cite{tfgnvf-popl-2016,gtnffvf-jfp-2019}.
The following benchmarks required changes:
\bmname{acquire}, \bmname{kcfa},
\bmname{snake}, \bmname{suffixtree}, \bmname{synth}, \bmname{take5},
\bmname{tetris}, and \bmname{zombie}.
%% FILL also quadT

In short, the trouble with adaptors and profiling is that the
name of the adaptor appears in contracts instead of the name of its clients.
If one adaptor has three clients, then profiling will attribute costs to one
library/adaptor boundary instead of the three library/client boundaries.
This kind of attribution is bad for the rational programmer because
it cannot modify the adaptor.
Only the library and client modules can get types.

The change is to add client-specific submodules to each adaptor.
Taking an adaptor with three clients as an example, the changes are:
\begin{enumerate}
  \item
    define generative types at the top level of the adaptor;
  \item
    export the generative types \emph{unsafely}, without any contract;
  \item
    create three submodules, one for each client, each of which imports
    the generative types and provides them safely; and
  \item
    modify the clients to import from the newly-created submodules rather
    than the top level.
\end{enumerate}
%
The submodules attach client-specific names to contracts.
They do not change run-time behavior.


\section{Skylines Per Benchmark}
\label{s:bm-sky}

Whereas~\cref{f:strategy-overall} reports the overall success rate for every
scenario in the experiment,~\cref{f:skyline:bm} separates the results
by benchmark.
Thus there are \numgtp{} plots.
Because the benchmarks vary in size from 4 to 10 modules,
each plot covers a distinct number of scenarios.
Refer to~\cref{t:blackhole} to see how many scenarios each
benchmark has.
The colors are from a colorblind-friendly palette~\cite{w-nature-2011}.

\paragraph{Observations}

\begin{itemize}
  \item
    The plots for \bmname{mbta}, \bmname{dungeon}, and \bmname{take5}
    are empty because none of their scenarios can reach a 1x configuration~(\cref{t:blackhole}).

  \item
    The plot for \bmname{synth} is similar to the overall picture~(\cref{f:strategy-overall})
    because \bmname{synth} has many more scenarios than the other benchmarks.
    Despite this imbalance, most benchmarks agree with the overall picture.
    The results for \bmname{forth}, \bmname{suffixtree}, \bmname{fsm},
    \bmname{fsmoo}, \bmname{snake}, \bmname{zombie}, \bmname{tetris}, and \bmname{jpeg}
    all confirm the superiority of the optimistic boundary strategy.

  \item
    Both \bmname{morsecode} and \bmname{lnm} do better with statistical profiles
    than with boundary profiles.
    These benchmarks have relatively low overhead in their configurations.
    Boundary profiling therefore reports no information, whereas the statistical
    profiler can make progress.
    Curiously, statistical (total) beats statistical (self) in
    \bmname{morsecode} and the reverse is true in \bmname{lnm}.

  \item
    All three profilers do well in \bmname{kcfa}.
    Feature-specific is best, but only by a small margin.

\end{itemize}

\begin{figure}[t]
  \newcommand{\kkrow}[1]{\bmname{#1} \\ \includegraphics[width=0.39\columnwidth]{data/sky/#1-feasible.pdf}}
  \begin{tabular}[t]{ll}
    \begin{tabular}[t]{l}
      \kkrow{morsecode} \\
      \kkrow{forth} \\
      \kkrow{fsm} \\
      \kkrow{fsmoo} \\
      \kkrow{mbta} \\
      \kkrow{zombie} \\
      \kkrow{dungeon} \\
      \kkrow{jpeg} \\
    \end{tabular}
    &
    \begin{tabular}[t]{l}
      \kkrow{lnm} \\
      \kkrow{suffixtree} \\
      \kkrow{kcfa} \\
      \kkrow{snake} \\
      \kkrow{take5} \\
      \kkrow{acquire} \\
      \kkrow{tetris} \\
      \kkrow{synth}
    \end{tabular}
  \end{tabular}

  \caption{How scenarios in each benchmark does each strategy succeed in?}
  \label{f:skyline:bm}
\end{figure}


\section{Head to Head Per Benchmark}

\Cref{f:h2h:bm} compares the \emph{optimistic, boundary} strategy
against all the others in each benchmark.
Overall these plots support the conclusions from~\cref{s:hh}.

\paragraph{Observations}
\begin{itemize}
  \item
    The plots for \bmname{mbta}, \bmname{dungeon}, and \bmname{take5} are
    empty. No strategy ever succeeds.
  \item
    The plot for \bmname{acquire} is nearly empty, again because successes are rare.
    The conservative and configuration-aware strategies with boundary
    profiles are slightly better than the rest.
  \item
    Because boundary profiling tends to get stuck in \bmname{morsecode}
    and \bmname{lnm} due to low overhead at boundaries, there are noticeable red
    bars for all strategies that use statistical profiles.
    Statistical out-performs optimistic.
    But \emph{null} also does well and even beats statistical in
    \bmname{morsecode}.
    This unexpectedly high success of null suggests that profiling is not needed
    for these benchmarks; better performance is close at hand with any change to the
    boundaries.

\end{itemize}

\begin{figure}[t]
  \newcommand{\hhrow}[1]{\bmname{#1} \\ \includegraphics[width=0.39\columnwidth]{data/h2h/#1.pdf}}
  \begin{tabular}[t]{ll}
    \begin{tabular}[t]{l}
      \hhrow{morsecode} \\
      \hhrow{forth} \\
      \hhrow{fsm} \\
      \hhrow{fsmoo} \\
      \hhrow{mbta} \\
      \hhrow{zombie} \\
      \hhrow{dungeon} \\
      \hhrow{jpeg} \\
    \end{tabular}
    &
    \begin{tabular}[t]{l}
      \hhrow{lnm} \\
      \hhrow{suffixtree} \\
      \hhrow{kcfa} \\
      \hhrow{snake} \\
      \hhrow{take5} \\
      \hhrow{acquire} \\
      \hhrow{tetris} \\
      \hhrow{synth}
    \end{tabular}
  \end{tabular}

  \caption{Optimistic vs. the rest, comparing strict successes in each benchmark.}
  \label{f:h2h:bm}
\end{figure}
